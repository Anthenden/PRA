
\newif\ifwhole

\wholetrue
% Ajouter \wholetrue si on compile seulement ce fichier

\ifwhole
 \documentclass[a4page,10pt]{article}
     \input{entete}
 \begin{document}
\input{enonce}
\fi

\begin{Def}
On dit qu'une immersion $f : S \to \R^3$ est infinit�simalement rigide si toute d�formation infinit�simale $g$ est un d�placement infinit�simale $g$ est un d�placement infinit�simal $g=a \times f+b$, $a,b \in \R^3$
\end{Def}

\begin{Lemme}
Une immersion $f :S \to \R^3$ est infinit�simalement rigide si, et seulement si, toute immersion totalement isotrope $\varphi : S \to Q$ relevant $f$, i.e. telle que $\pi \circ \varphi =f : S \to \R^3$ est telle que $\varphi_+$ est constante.
\end{Lemme}
\begin{proof}
$\Rightarrow : $ D�j� fait. \\
$\Leftarrow : $ Soit $\varphi =(f : 1 : g: -f\cdot g) : S \to Q$ telle que $df \cdot dg=0$ et $\varphi_+$ est constante. \\
Alors il existe $h :S \to \R^3$ telle que $dg=h \times df$. \\
$d\varphi=(df:0:dg:-df\dot g-dg\cdot f)=(df : 0 : h \times dg : -df \cdot(g+f\times h))$ (gr�ce � la formule du produit triple) \\
On en d�duit que $d\varphi=(h,g+f \times h)_+(df,0)$. Et donc $\varphi_+=(h,g+f \times h)_+$. 
Comme $\varphi_+$ est constante alors $h$ et $g+f \times h$ aussi, c'est-�-dire, $g=-h \times f+b$, $b \in \R^3$.
\end{proof}


\begin{Thm}[Darboux-Sauer]
La rigidit� infin�t�simale est projectivement invariante : si $f : S \to \R^3 \subset \R P^3$ est une immersion, et $A \in \PGL_4(\R)=Aut(\R P^3)$ est une transformation projective telle que $A \circ f(S) \subset \R^3$, alors $A \circ f$ est infinit�simalement rigide si, et seulement si, $f$ l'est.
\end{Thm}
\begin{proof}
On peut commencer par remarquer qu'il suffit de montrer qu'un sens car il suffit de consid�rer $A^{-1}$. \\
$\PGL_4(\R)$ agit sur $\R P^7$ par $A \cdot (u : v)=( Au : {}^t A^{-1} v)$ o� $u,v \in \R^4, A \in \PGL_4(\R)$. \\
Cette action laisse stable $Q$ : \\ $\forall (u:v) \in Q, \forall A \in \PGL_4(\R),q(A\cdot (u:v))=q(Au : {}^t A^{-1} v)=Au \cdot {}^t A^{-1} v=u\cdot v=0$. Elle laisse stable aussi $P_1$ et $P_2$. \\
Cette action commute avec $\pi_1 : Q \setminus P_2 \to P_1$ (sur $Q \setminus P_2$) : \\
$\forall (u:v) \in Q, \forall A \in \PGL_4(\R),\pi_1(A \cdot (u,v))=(Au : 0)=A(u \oplus 0)=A(\pi(u : v))$.
Par cons�quent, pour $A \in \PGL_4(\R)$ et $\varphi$ une immersion totalement isotrope relevant $f$ : $\pi_1(A \circ \varphi)=A(\pi_1(\varphi))=A \circ f$ i.e. $A \circ \varphi$ rel�ve $A \circ f$ \\ 
Comme, de plus, pour tout $p \in S$, $d_p(A \circ \varphi)=d_{\varphi(p)} A \circ d_p \varphi=A \circ d_p \varphi$, alors $\tau_p (A \circ \varphi)=A \tau_p \varphi$ et donc $A(\varphi_+)=(A \circ \varphi)_+$, ce qui est le r�sultat voulu. 

\end{proof}






\ifwhole
 \end{document}
\fi