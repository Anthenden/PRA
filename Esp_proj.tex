
\newif\ifwhole

\wholetrue
% Ajouter \wholetrue si on compile seulement ce fichier

\ifwhole
 \documentclass[a4page,10pt]{article}
     \input{entete}
 \begin{document}
\input{enonce}

\fi
\section{Espaces projectifs}
On consid�rera ici seulement des espaces projectifs sur $\R$ mais on peut en d�finir pour n'importe quel corps (non n�cessairement commutatifs).
\begin{Def}
Soit $E$ un $\R$-espace vectoriel non r�duit � $\{0\}$. \\
On appelle espace projectif associ� � $E$ l'espace quotient $E \setminus \{0\}/\Rcal$ o� $x \Rcal y \Leftrightarrow \exists \lambda \in \R^*,y=\lambda x$. On le note $P(E)$ 
\end{Def}
On peut voir $P(E)$ comme l'ensemble des droites vectorielles de $E$. En effet, les classes d'�quivalence de $\Rcal$ sont ces droites vectorielles.

Si $E=\R^{n+1}$, l'espace projectif $P(\R^{n+1})$ est not� $P^{n}(R)$ ou $\R P^{n}$.
\begin{Not}La classe d'�quivalence de $(x_0,\ldots,x_n)$ est not� $[x_0,\ldots,x_n]$.\end{Not}

\subsection{ Topologie de l'espace projectif}
On munit $\R P^n$ de la topologie quotient, c'est-�-dire que $U$ est un ouvert de $\R P^n$ si $\pi^{-1}(U)$ est un ouvert de $\R^{n+1}\setminus \{0\}$ o� $\pi : \R^{n+1} \setminus \{0\} \to \R P^n$ est la projection canonique.

\begin{Prop}
$\R P^n$ est compact et connexe par arcs.
 \end{Prop}
\begin{proof}
Comme $\pi(S^n)=\R P^n$ et que la sph�re $S^n$ est (quasi-)compacte et connexe par arcs. alors $\R P^n$ est quasi-compact et connexe par arcs.
Il faut montrer que $\R P^n$ est s�par�. 
Soient $x \neq y \in \R P^n$ et $\alpha$ et $\beta$ tels que $|\alpha|=|\beta|, x=\pi(\alpha)$ et $y=\pi(\beta)$ 
On remarque que $U=\pi(B(\alpha,|\alpha-\beta|/3))\ni x$ et $V=\pi(B(\beta,|\alpha-\beta|/3)) \ni y$ sont disjoints
\end{proof}
On peut aller un peu plus loin en montrant que $\R P^{n}$ est hom�omorphe � $S^n/\{\pm Id\}$ % peut-�tre diff�omorphe ?
%D�mo � faire (peut-�tre...)
\subsection{Structure de vari�t� de l'espace projectif}
Soient $U_i=\{[x_0:\ldots:x_n] \in \R P^n | x_i \neq 0\}$, $0 \leq i \leq n$, des ouverts de $\R P^n$ et $\varphi_i : [x_0:\ldots:x_n] \in U_i \mapsto \left(\frac{x_0}{x_i}, \ldots, \widehat{\frac{x_i}{x_i}},\ldots,\frac{x_n}{x_i}\right)$.

\begin{Thm}
$(U_i,\varphi_i)_{i \in I}$ forme un atlas de $\R P^n$.
\end{Thm}
\begin{proof}
Soit $i \in [\![0,n]\!]$. \\
$\varphi_i$ est une bijection de r�ciproque $\varphi_i^{-1} : (x_0,\ldots,\widehat{x_i},\ldots,x_n) \in \R^n \mapsto [x_0,\ldots,1, \ldots,x_n] \in U_i$ (le $1$ est en $i$�me position).
$\forall (x_0,\ldots,x_n) \in \pi^{-1}(U_i), \varphi_i \circ \pi(x_0,\ldots,x_n)=\left(\frac{x_0}{x_i}, \ldots, \widehat{\frac{x_i}{x_i}},\ldots,\frac{x_n}{x_i} \right)$. On en d�duit que $\varphi_i \circ \pi$ est continue et donc $\varphi_i$ aussi. \\
$\varphi_i^{-1}=\pi(dx_0,\ldots,1,\ldots,dx_n)$ o� $dx_j(x_0,\ldots,\widehat{x_i},\ldots,x_n)=x_j$, $i \neq j$. On en d�duit que $\varphi_i^{-1}$ est aussi continu et donc $\varphi_i$ est un hom�omorphisme. \\
De plus, on voit que, pour tout $i \neq j$, $\varphi_i \circ \varphi_j^{-1}(x_0,\ldots,x_n)=\left(\frac{x_0}{x_i},\ldots,\frac{x_{i-1}}{x_i},\frac{x_{i+1}}{x_i},\ldots, \frac{1}{x_i},\ldots,\frac{x_n}{x_i}\right)$ pour $(x_0,\ldots,x_n) \in\varphi_j(U_i \cap U_j)$ et donc que $\varphi_i \circ \varphi_j^{-1}$ est $\Ccal^\infty$. 
\end{proof}
\begin{Cor}
Muni de l'atlas $(U_i,\varphi_i)_{i \in I}$, $\R P^n$ est une vari�t� diff�rentielle.
\end{Cor}




\ifwhole
 \end{document}
\fi