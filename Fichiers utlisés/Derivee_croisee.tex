
\newif\ifwhole

\wholetrue
% Ajouter \wholetrue si on compile seulement ce fichier

\ifwhole
 \documentclass[a4page,10pt]{article}
     \input{entete}
 \begin{document}
\input{enonce}

\fi

\begin{Lemme}\label{deriv_croisee}
Soit $(f,g,h)$ un triplet de Darboux. Alors $\partial_1 h \times \partial_2 f=\partial_2 h \times \partial_1 f$ 
(dans des coordonn�es locales de $S$).
\end{Lemme}

\begin{proof}
En diff�renciant la 1-forme $dg=h \times df=(h \times \partial_1 f)dx+(h \times \partial_2 f)dy$, on obtient : \\
$0=(\partial_1 (h \times \partial_1 f)dx + \partial_2 (h \times \partial_1 f) dy )dx+(\partial_1 (h \times \partial_2 f)dx + \partial_2 (h \times \partial_2 f) dy )dy$. \\
En examinant la partie en $dx \wedge dy$ et en utilisant le lemme de Schwarz, on en d�duit le r�sultat.
\end{proof}

On peut d�composer $\partial_i h$, $i=1,2$,dans la base $\partial_1 f,\partial_2 f,v_f$ (o� $v_f$ engendre $Im(df)^\perp$) : $\partial_i h=\alpha_i \partial_1 f+\beta_i \partial_2 f+\gamma_i v_f$. En prenant le produit scalaire par $\partial_1 f \times \partial_2 f$ des deux c�t�s puis en utilisant le lemme pr�c�dent (et le produit triple), on obtient que $\gamma_i=0$, $i=1,2$ c'est-�-dire $\partial_i h \in Im(df)$. De ce fait, $Im(dh) \subset Im(df)$.





\ifwhole
 \end{document}
\fi