
\newif\ifwhole

\wholetrue
% Ajouter \wholetrue si on compile seulement ce fichier

\ifwhole
 \documentclass[a4page,10pt]{article}
     \input{entete}
 \begin{document}
\input{enonce}
\fi

\begin{Ex}
Supposons que $q$ soit la forme quadratique de signature $(n,0)$ i.e. $q(x_1,\ldots,x_n)=\sum_{i=1}^n x_i^2$. \\
On peut identifier $\R^n$ � un sous-ensemble de $Q=\{ [s : x :t ] | q(x)=st  \} \subset \R P^{n+1}$ (on notera $N$ la forme quadratique d�finie par $N(s,x,t)=q(x)-st)$. \\
Montrons que l'on peut identifier $Q$ � la sph�re unit� de $\R^{n+1}$. \\
Soient $H=\{(s,x,t) \in \R \times \R^n \times \R | s+t=0\}$ un hyperplan de $\R^{n+2}$ et $\Bcal=\{e_0-e_{n+1},e_2,\ldots,e_n,e_0+e_{n+1}\}$ une base de $\R^{n+2}$ o� $(e_0,\ldots,e_{n+1})$ est la base canonique de $\R^{n+2}$. Ainsi, dans cette base, $H=\{ (y_0,\ldots,y_{n+1})_\Bcal | y_{n+1} =0\}$ et donc, $\R P^{n+1} \setminus H=\{[y_0 : \ldots : y_{n+1}]_\Bcal | y_{n+1} \neq 0\}$. On a une identification naturelle de $\R P^{n+1} \setminus H$ avec $\R^{n+1}$ gr�ce � l'isomorphisme $\varphi : [y_0 : \ldots : y_{n+1}] \mapsto (y_0/y_{n+1} , \ldots , y_{n}/y_{n+1})$. \\
Si $y = y_0(e_0-e_{n+1})+\sum_{i=1}^n y_i e_i+y_{n+1} (e_0-e_{n+1}) \in Q$ alors 
	\[0=N(y)=q(y_1,\ldots,y_n)-(y_0+y_{n+1})(y_0-y_{n+1})=\sum_{i=0}^{n+1} y_i^2-y_{n+1}^2
\]
La r�ciproque est �videmment vraie. \\
Comme $Q$ est inclus dans $\R P^{n+1} \setminus H$ (car dans le cas contraire, on aurait des points $(s,x,t) \neq 0$ tels que $q(x)=-s^2<0$) ainsi gr�ce � notre identification de $\R P^{n+1} \setminus H$ avec $\R^{n+1}$, on obtient que $Q$ s'identifie avec la sph�re unit�.
La r�ciproque de l'isomorphisme $\varphi$ est d�finie de la fa�on suivante : $\varphi^{-1}(x,s)=[ s : x : 1]_\Bcal$ (dans la base $\Bcal$) et donc $\varphi^{-1}(x,s)=[ s+1 : x : s-1]=[\frac{s+1}{s-1} : x : 1]$. Ainsi, la projection $\pi : [s : x:1] \in Q \setminus \{p\} \mapsto x \in \R^n$ s'identifie, gr�ce � l'isomorphisme $\varphi$, � la projection st�r�ographique de $S^n \setminus N$ sur $\R^n$
\end{Ex}





\ifwhole
 \end{document}
\fi