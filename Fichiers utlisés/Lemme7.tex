
\newif\ifwhole

\wholetrue
% Ajouter \wholetrue si on compile seulement ce fichier

\ifwhole
 \documentclass[a4page,10pt]{article}
     \input{entete}
 \begin{document}
\input{enonce}
\fi


Soit $a \in \Ocal$. Posons $\Re(a)=\left\langle a,1\right\rangle$ et $\overline{a}=2\Re(a)-a$. 

Cette conjugaison prolonge celle de $\H$ car la restriction de $\left\langle \cdot,\cdot\right\rangle$ est la norme euclidienne sur $\H$.
En faisant les calculs, on obtient que $\overline{(a,b)}=(\overline{a},-b)$ pour tout $(a,b) \in \Ocal$

\begin{Lemme}
$a \mapsto \overline{a}$ est un anti-automorphisme involutif de $\Ocal$.
\end{Lemme}
\begin{proof}
Par bilin�arit� de $\left\langle \cdot,\cdot\right\rangle$, la conjugaison est lin�aire. \\
Soit $a \in \Ocal$. \\
$\overline{\overline{a}}=2\left\langle \overline{a},1\right\rangle-\overline{a}=2\left\langle 2\left\langle a,1\right\rangle-a ,1\right\rangle-2\Re(a)+a$. \\
Comme $\left\langle 1,1\right\rangle=0$ alors, $\overline{\overline{a}}=a$.
Montrons que la conjugaison renverse les produits : \\
Soient $a=(x,y),b=(x',y') \in \Ocal$. \\
$\overline{ab}=(\overline{x'}\overline{x}+y\overline{y'},-\overline{x}y'-\overline{y}\overline{x'})=\overline{b}\;\overline{a}$
\end{proof}

\begin{Lemme}
Soit $a \in \H$. $L_a,L_{\overline{a}}$ (resp. $R_a,R_{\overline{a}}$) sont adjoints. 
\end{Lemme}
\begin{proof}
Soit $x,y \in \Ocal$. \\
En polarisant $\left\langle L_a(x),y \right\rangle=\left\langle ax,y \right\rangle$, on obtient : \\
	\[\left\langle L_a(x),y \right\rangle=\frac{N(ax+y)^2-N(ax)^2-N(y)^2}{2}
\]
Puis, par multiplicativit� de $N$, on a : 
\[\left\langle L_a(x),y \right\rangle=|a|^2\frac{N(x+a^{-1}y)^2-N(x)^2-N(a^{-1}y)^2}{2}
\]
\[\left\langle L_a(x),y \right\rangle=|a|^2 \left\langle x,a^{-1}y\right\rangle=\left\langle x,\overline{a}y\right\rangle=\left\langle x,L_{\overline{a}}(y)\right\rangle
\]
\end{proof}
\begin{Prop}
Soit $a \in \Ocal$. $L_a,L_{\overline{a}}$ (resp. $R_a,R_{\overline{a}}$) sont adjoints. 
De plus, $L_a \circ L_{\overline{a}}=N(a)Id=R_{\overline{a}} \circ R_a$.
\end{Prop}
\begin{proof}
Soit $a=a_1+\ell a_2 \in \Ocal$ et $x=x_1+\ell x_2,y \in \Ocal$. \\
En utilisant le lemme pr�c�dent, $\ell x_1=\overline{x_1} \ell$ puis que $\ell x=\ell x_1+x_2$ (et donc que $\left\langle x, \cdot \right\rangle=-\left\langle \ell x, \cdot \right\rangle$, on obtient : 
$\left\langle ax,y\right\rangle=\left\langle a_1x,y\right\rangle+\left\langle \overline{a_2}\ell x,y\right\rangle=\left\langle x,\overline{a_1}y\right\rangle+\left\langle \ell x,a_2y\right\rangle=\left\langle x,\overline{a_1}y\right\rangle+\left\langle x,-a_2y\right\rangle=\left\langle x,\overline{a}y\right\rangle$.
Gr�ce aux identit�s \ref{}, on a : $a(\overline{a}x)=N(a)x=(x \overline{a})a$ et donc $L_a \circ \L_{\overline{a}}=N(a) Id=R_{\overline{a}}\circ R_a$
\end{proof}

\begin{Prop}
Soit $a,b \in \Ocal \setminus \{0\}$ tels que : $ba=0$. Alors, $a (\Ocal \overline{b})=\R \overline{b}$ et $(\overline{a} \Ocal b)=\R \overline{a}$
\end{Prop}

\begin{proof}
Soit $x \in \Ocal$. Le r�sultat d�coule des �quations \ref{}:  
	\[a(x \overline{b})=2\left\langle a,\overline{x}\right\rangle \overline{b}-x\underbrace{(\overline{a} \overline{b})}_{=0} 
\]
et 
\[(\overline{a}x)b=2\left\langle x,b\right\rangle \overline{a}-\underbrace{(\overline{a} \overline{b})}_{=0}x 
\]

\end{proof}
















\ifwhole
 \end{document}
\fi