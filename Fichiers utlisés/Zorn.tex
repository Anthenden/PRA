
\newif\ifwhole

\wholetrue
% Ajouter \wholetrue si on compile seulement ce fichier

\ifwhole
 \documentclass[a4page,10pt]{article}
     \input{entete}
 \begin{document}
\input{enonce}
\fi



Afin de faire des calculs de mani�re plus simple avec les octonions d�ploy�s , on va consid�rer le mod�le des matrices de Zorn d�fini de la fa�on suivante : \\
La $\R$-alg�bre (unitaire, non commutative, non associative) des matrices de Zorn est $\Zcal=\left\{ \begin{pmatrix} a & x \\ y & b \end{pmatrix} | a,b \in \R, x,y \in \R^3   \right\}$ muni de la somme et la multiplication par un scalaire composantes par composantes  i.e. 
	\[\begin{pmatrix} a & x \\ y & b \end{pmatrix}+\lambda \begin{pmatrix} a' & x' \\ y' & b' \end{pmatrix}=\begin{pmatrix} a+\lambda a' & x+\lambda x' \\ y+\lambda y' & b+\lambda b' \end{pmatrix}
	\]
	et du produit : \\
	\[\begin{pmatrix} a & x \\ y & b \end{pmatrix} \times \begin{pmatrix} a' & x' \\ y' & b' \end{pmatrix}=\begin{pmatrix} aa'+x \cdot y' & ax'+b'x-y \times y' \\ ay'+by'+x \times x' & bb'+x' \cdot y \end{pmatrix}.
	\] 
L'alg�bre des matrices de Zorn est isomorphe � celle des octonions d�ploy�s gr�ce � l'application $\varphi : z=(a+x,b+y) \in \Ocal \mapsto \begin{pmatrix} a+b & x+y \\ -x+y & a-b \end{pmatrix} \in \Zcal$ o� on a identifi� $bi+cj+dk$ avec le vecteur $(b,c,d)$. \\
On peut transporter la (pseudo)-norme de $\Ocal$ � $\Zcal$ en posant $N(\varphi(z))=q(z)$ pour $z \in \Ocal$. On obtient alors que $N$ est le "d�terminant" de $\Zcal$ i.e. $N\left(\begin{pmatrix} a & x \\ y & b \end{pmatrix}\right)=ab-x\cdot y$. 

On va maintenant traduire les espaces $(u,v)_-$ et $(u,v)_+$ avec les matrices de Zorn : \\ %A r��crire
Tout d'abord, rappelons que $(u,v)_-=\{ (x,s,y,t) | x \in \R^3,t \in \R,y=u\times x+vt, s=-v \cdot x\}$. De ces deux �galit�s, on peut en obtenir deux autres : $su-(u \cdot v)x-y \times v=0$ et $y \cdot u+tu \cdot v=0$ i.e. \\$(x,s,y,t) \in (u,v)_-$ si, et seulement si, $\begin{pmatrix} s & x \\ y & -t \end{pmatrix} \times \begin{pmatrix} 1 & u \\ v & -u \cdot v \end{pmatrix}=0$.\\
 De la m�me fa�on,\\ $(x,s,y,t) \in (u,v)_-$ si, et seulement si, $\begin{pmatrix} u \cdot v & -u \\ -v & 1 \end{pmatrix}\times\begin{pmatrix} s & x \\ y & -t \end{pmatrix} =0$ \\
(on peut remarquer que la matrice $\begin{pmatrix} u \cdot v & -u \\ -v & 1 \end{pmatrix}$ est la matrice conjugu�e (par transport de celle de $\Ocal$) de $\begin{pmatrix} 1 & u \\ v &-u \cdot v  \end{pmatrix}$). \\
Plus g�n�ralement, on peut identifier les �l�ments $(x:s:y:t)$ de $Q$ aux annulateurs � gauche (resp. � droite) de $Z=\begin{pmatrix} s & x \\ y & -t \end{pmatrix}$ dans $Q_+$ (resp. $Q_-$) (on remarque $N(Z)=0$ et donc ce sont bien des SETIM).On notera ces annulateurs $\Ker(L_Z)$ (resp. $\Ker(R_Z)$) o� $L_Z$ est la multiplication � gauche par $Z$ et $R_Z$ est la multiplication � droite par $Z$. \\
On a l'analogue du th�or�me \ref{} dans l'alg�bre des octonions d�ploy�s : 
\begin{Thm}
La relation d'incidence $\dim(\Ker(L_{Z_+}) \cap \Ker(R_{Z_-}))=3$ entre $Q_+$ et $Q_-$ �quivaut � $Z_-Z_+=0$
\end{Thm}
 




\ifwhole
 \end{document}
\fi 