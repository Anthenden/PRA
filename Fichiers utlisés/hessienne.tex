
\newif\ifwhole

\wholetrue
% Ajouter \wholetrue si on compile seulement ce fichier

\ifwhole
 \documentclass[a4page,10pt]{article}
     \input{entete}
 \begin{document}
\input{enonce}
\fi
\begin{Def}
Soit $M$ une vari�t� diff�rentielle. \\
Un champ de vecteurs sur $M$ est une application $X : U \subset M \to TM$ telle que $\forall x \in U,X(x)\in T_xX$.
\end{Def}
\begin{Def}
On associe � un champ de vecteur $X$ sur $M$ la d�rivation $L_X : f \in \Ccal^\infty(M)  \mapsto (x \mapsto X\cdot f(x):=d_xf(X(x))) \in \Ccal^\infty(M)$.
\end{Def}
\begin{Def}
On appelle crochet de deux champs de vecteurs $X$ et $Y$ le champ de vecteur associ� � $L_X \circ L_Y-L_Y \circ L_X$.
\end{Def}

\begin{Defthm}
Soit $f : M \to \R$ une application lisse. Soit $a \in M$ tel que $d_af=0$. Soit $u,v \in T_aM$ et $X$ et $Y$ deux champs de vecteurs tels que $X(a)=u$ et $Y(a)=v$. \\
Posons $H_af(u,v):=\left[X.df(Y)\right](a)$.
Cette valeur ne d�pend pas des champs de vecteurs $X$ et $Y$ choisis et l'application $H_af:(u,v) \mapsto H_af(u,v)$ est une forme bilin�aire sym�trique de $T_aM$.
\end{Defthm}
\begin{proof}
Montrons, tout d'abord, que $\left[X.df(Y)\right](a)=\left[Y.df(X)\right](a)$ $(*)$ : \\
$\left[X.df(Y)-Y.df(X)\right](a)=\left[X.df(Y)-Y.df(X)\right](a)=\left[X.L_Y(f)-Y.L_X(f)\right](a)=\left[L_X \circ L_Y-L_Y \circ L_X\right](f)(a)=[X,Y].f(a)=d_af([X,Y](a))=0$. \\
Le terme � gauche de $(*)$, comme fonction de $X$,  ne d�pend que de $X(a)=u$ et de m�me, le terme � droite de $(*)$,comme fonction de $Y$, ne d�pend que de $Y(a)=v$.
On en d�duit que $H_af(u,v)$ ne d�pend pas ni de $X$ ni de $Y$ mais seulemenent de $u$ et $v$. \\
Avec $(*)$, on voit que $H_af$ est sym�trique. La bilin�arit� est imm�diate.
\end{proof}
L'application $H_af$ est appel� hessienne de $f$ en $a$.







\ifwhole
 \end{document}
\fi