
\newif\ifwhole

\wholetrue
% Ajouter \wholetrue si on compile seulement ce fichier

\ifwhole
 \documentclass[a4page,10pt]{article}
     \input{entete}
 \begin{document}
\input{enonce}

\fi


\begin{Thm}
Soit $F \subset \R^4 \oplus \R^4$ un sous-espace vectoriel totalement isotrope de dimension $k \leq 4$. Alors $E$ est le graphe d'une application orthogonale $\widetilde{E} \to \R^4$ o� $\widetilde{F}$ est un sous-espace vectoriel de $\R^4$ de dimension $k$. \\
R�ciproquement, tout graphe d'une application orthogonale est totalement isotrope.   
\end{Thm}

\begin{proof}
Soient $x \oplus y, x \oplus y' \in F$ Alors leur diff�rence $0 \oplus (y-y')$ est dans $F$. Par cons�quent, $|y-y'|=0$ et donc $y=y'$. \\
On peut donc d�finir une application $f : \widetilde{E} \subset \R^4 \to \R^4$ d�finie de la fa�on suivante : \\
Si $x \oplus y \in F$ alors $f(x):=y$.  \\
On peut, de plus, remarquer que $\widetilde{E}$ est un espace vectoriel et que $f$ est une application lin�aire orthogonale (car $|x|=|f(x)|$, par orthogonalit� de $F$) injective et dont le graphe est $F$ (par construction). Par cons�quent, $k=\dim(F)=\rg(f)=\dim(\widetilde{E})$. \\
R�ciproquement, si $Q$ est une application orthogonale, $\forall x \in \widetilde{E}, q(x,Q(x))=|x|-|Q(x)|=0$
\end{proof}
En particulier, les sous-espaces vectoriels totalement isotropes maximaux sont les graphes des �l�ments de $O(4)$.

\begin{Cor}
Soit $V$ un sous-espace totalement isotrope de dimension $3$ de $\R^4 \oplus \R^4$. Il existe deux sous-espaces totalement isotropes maximaux contenant $V$.
\end{Cor}
\begin{proof}
Soient $f : \widetilde{E} \to \R^4$ l'application orthogonale dont le graphe est $V$ et $\Bcal$ une base orthonorm�e de $\widetilde{E}$. \\
Il existe deux vecteurs compl�tant $\Bcal$ en une base orthonorm�e de $\R^4$. En effet, $\widetilde{E}^\perp$ est de dimension 1 et donc a deux g�n�rateurs unitaires oppos�s. $f$ a deux prolongements en une application orthogonale de $O(4)$, un dans $\SO(4)$ et un dans $O^-(4)$.
\end{proof}
\begin{Cor}
$V^\perp/V$ est de dimension 2 et a deux droites totalement isotropes.
\end{Cor}




\ifwhole
 \end{document}
\fi