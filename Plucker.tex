
\newif\ifwhole

\wholetrue
% Ajouter \wholetrue si on compile seulement ce fichier

\ifwhole
 \documentclass[a4page,10pt]{article}
     \input{entete}
 \begin{document}
\input{enonce}
\fi

\section{Alg�bre ext�rieure}
Soit $E$ un espace vectoriel de dimension finie sur un corps commutatif $k$. \\
On appelle $p$�me puissance ext�rieure de $E$ l'espace vectoriel $\bigwedge^p E$,quotient de $E^{\otimes p}$ par le sous-espace vectoriel $\Zcal_p=\Vect(x_{1} \otimes \ldots \otimes x_{p} | x_i \in E,\exists m \neq n, x_{m}=x_{n})$ (par convention, $\bigwedge^0 E=k$). \\
On note $x_1 \wedge \ldots \wedge x_p$ la classe d'�quivalence $x_1 \ldots \ldots \ldots x_p$. \\
Le passage au quotient conserve les propri�t�s d'associativit�e et de $p$-lin�arit� et fait que $\forall u,v \in E, u \wedge v=-v \times u$ \\
On appelle alg�bre ext�rieure la somme directe des puissances ext�rieures : $\bigwedge^* E=\bigoplus_{p \in \N} \bigwedge^p E$. On munit $\bigwedge^* E$ d'une structure d'alg�bre gradu�e gr�ce au produit : $(x_1 \wedge \ldots x_p,y_1 \wedge \ldots y_q) \in \bigwedge^p E \times \bigwedge^q E \mapsto x_1 \wedge \ldots x_p \wedge y_1 \wedge \ldots y_q \in \bigwedge^{p+q} E$.


\begin{Lemme}
$\Zcal_p=\Vect(x_{1} \otimes \ldots \otimes x_{p} | x_i \in E, \{x_1,\ldots,x_p\} \text{ est une famille li�e })$
\end{Lemme}
\begin{proof}
Posons $V=\Vect(x_{1} \otimes \ldots \otimes x_{p} | x_i \in E, \{x_1,\ldots,x_p\} \text{ est une famille li�e })$. \\
On a clairement $\Zcal_p \subset V$. R�ciproquement, soit $x_1 \wedge \ldots \wedge x_p \in V$ i.e.  $\{x_1,\ldots,x_p\}$ est li�e. On a donc un $i\in [\![1,n]\!]$ et des $\lambda_j \in k,j \neq i$ tel que $x_i=\sum_{j \neq i} \lambda_jx_j$ et donc : \\
 $x_1 \wedge \ldots \wedge x_i \wedge \ldots \wedge x_p=\sum_{j \neq i} \lambda_j x_1 \wedge \ldots \wedge x_j \wedge \ldots \wedge x_p \in \Zcal_p$ 
\end{proof}
Cela nous permet de voir que $\bigwedge^p E= \{0\}$ si $p>dim(E)$

\begin{Prop}
Soit $\Bcal=\{e_1,\ldots,e_n\}$ une base de $E$. Pour $p \leq n$,$\{e_{i_1} \wedge \ldots \wedge e_{i_p},1 \leq i_1<\ldots <i_p \leq n\}$ est une base de $\bigwedge^p E$.
\end{Prop}
\begin{proof}
Comme la famille $\{e_1 \otimes \ldots \otimes e_p|e_i \in \Bcal\}$ est une base de $E^{\otimes p}$ alors le passage au quotient fait que l'espace quotient a comme base $ \{e_1 \wedge \ldots \wedge e_p|e_i \in \Bcal, i \neq j \Rightarrow e_i \neq e_j\}$. En ordonnant les $e_i$; on obtient le r�sultat. 
\end{proof}

\begin{Cor}
$\dim_k \bigwedge^p E=\binom{\dim E}{p}$
\end{Cor}
En particulier, $\dim_k \bigwedge^{\dim E} E=1$ (cela permet de d�finir le d�terminant)

\begin{Cor}
Soient $\{e_1,\ldots,e_m\},\{f_1,\ldots,f_m\}$ deux familles libres d'�l�ments de $E$. Alors $\Vect(e_1,\ldots,e_m)=\Vect(f_1,\ldots,f_m)$ si, et seulement si, $e_1 \wedge \ldots \wedge e_m$ et $f_1 \wedge \ldots \wedge f_m$ sont colin�aires.
\end{Cor}

\begin{proof}
$\Rightarrow$ : Soit $W=\Vect(e_1,\ldots,e_m)$. Alors $e_1 \wedge \ldots \wedge e_m$, $f_1 \wedge \ldots \wedge f_m \in \bigwedge^k W$. Ce dernier ensemble �tant de dimension 1 et donc $e_1 \wedge \ldots \wedge e_m$ et $f_1 \wedge \ldots \wedge f_m$ sont colin�aires.\\
$\Leftarrow$ : Soit $W=\Vect(e_1,\ldots,e_m)$ et $W'=\Vect(f_1,\ldots,f_m)$. Si $f_i \notin W$ alors on compl�te $\{f_i,e_1,\ldots,e_m\}$ en une base $\{v_1,\ldots,v_n\}$ de $E$.
Pour donner les id�es, prenons $i=1$, \\
$\lambda e_1 \wedge \ldots \wedge e_m=f_1 \wedge f_2 \wedge \ldots \wedge f_n=f_1 \wedge \sum_{j_2} a_{2j_2} v_{j_2} \wedge \ldots \wedge \sum_{j_m} a_{mj_m} v_{j_m}=\sum_{j_2}\cdots\sum_{j_m}f_1 \wedge  a_{2j_2} v_{j_2} \wedge \ldots \wedge  a_{mj_m} v_{j_m}$.
A droite de l'�galit�, $f_1$ apparait partout alors qu'� droite, il n'intervient pas. Ce qui est absurde car $\lambda \neq 0$.
\end{proof}

\section{Plongement de Pl�cker}
\begin{Def}[Plongement de Pl�cker]
On appelle plongement de Pl�cker l'application $\psi_{r,n} : G(r,n) \to P(\bigwedge^r k^n)$ qui envoie un sous-espace vectoriel $W$ de dimension $r$ de $k^n$ de base $e_1,\ldots,e_r$ sur $[e_1 \wedge \ldots \wedge e_r]$
\end{Def}
\begin{Cor}
$\psi_{r,n}$ ne d�pend pas du choix de la base et est injectif.
\end{Cor}

Nous allons maintenant d�terminer des �quations polynomiales v�rifi�es par $\psi_{r,n}(G(r,n))$.

\begin{Lemme}
Soit $W \in G(r,n)$ et $\{e_1,\ldots,e_r\}$ une base de $W$. Alors $W=\{x \in k^n |x \wedge (e_1 \wedge \ldots \wedge e_r)=0\}$ 
\end{Lemme}
\begin{proof}
Posons $E_W=\{x \in k^n |x \wedge (e_1 \wedge \ldots \wedge e_r)=0\}$.
On remarque ais�ment que $W$ est un espace vectoriel et que $W \subset E_W$. \\
Soit $x \in E_W$ ie $x \wedge (e_1 \wedge \ldots \wedge e_r)=0$ et donc $\{x,e_1,\ldots,e_r\}$ est une famille li�e. Comme $\{e_1,\ldots,e_r\}$ est une base alors $x \in \Vect(e_1,\ldots,e_r)=W$
\end{proof}
\begin{Lemme}
Soit $\Lambda \in \bigwedge^r k^n$ tel que $W=\{ x \in k^n | \Lambda \wedge x=0\}$ soit de dimension $r$. Alors $\psi_{r,n}(W)=[\Lambda]$
\end{Lemme}
\begin{proof}
%Pour montrer ce lemme, on peut tout d'abord remarquer que l'action de $\GL_n(k)$ commute avec $\psi_{r,n}$. $\GL_n(k)$ agit sur $P(\bigwedge^r k^n$ par $P \cdot [e_1 \wedge \ldots \wedge e_r]=[Pe_1 \wedge \ldots \wedge Pe_r]$. De plus, comme $\{Pe_1,\ldots, Pe_r\}$ est une base de $W$ alors $\psi_{r,n}(PW)=P\cdot \psi_{r,n}(W)$. \\
Soit $\{e_1,\ldots,e_r\}$ une base de $W$. On peut compl�ter cette base en un base $\{e_1,\ldots,e_n\}$ de $k^n$. \\
On peut d�composer $\Lambda$ sur la base associ�e de $\bigwedge^r k^n$ : \\
$\Lambda=\sum_p \lambda_p e_{i_{1p}} \wedge \ldots \wedge e_{i_{rp}}$. \\
Comme $\Lambda \wedge e_i=0$ alors $\Lambda=\sum_{p,i_{jp} \neq 0} \lambda_p e_1 \wedge e_{i_{1p}} \wedge \ldots \wedge e_{i_{rp}}$. \\
Comme les $e_1 \wedge e_{i_{1p}} \wedge \ldots \wedge e_{i_{rp}}$ forment une famille libre alors les $\lambda_p$ correspondant sont nulles.
En it�rant ce processus, on trouve que les scalaires correspondant aux multi-vecteurs contenant $e_j$ pour $j>r$ sont nuls et donc $\Lambda=\lambda e_1 \wedge \ldots \wedge e_r$, ce qu'il fallait d�montrer.
\end{proof}






\ifwhole
 \end{document}
\fi