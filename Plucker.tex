
\newif\ifwhole

\wholetrue
% Ajouter \wholetrue si on compile seulement ce fichier

\ifwhole
 \documentclass[a4page,10pt]{article}
     \usepackage[Latin1]{inputenc}
\usepackage[francais]{babel}
\usepackage{amsmath,amssymb,amsthm}
\usepackage{textcomp}
\usepackage{mathrsfs}
\usepackage{algcompatible,algorithm  }
\usepackage[all]{xy}
\usepackage{hyperref}
\usepackage{fancyhdr}
\usepackage{supertabular}
\pagestyle{plain}

\toks0=\expandafter{\xy}
\edef\xy{\noexpand\shorthandoff{!?;:}\the\toks0 }
\makeatletter

\renewcommand*{\ALG@name}{Algorithme}

\makeatother

\renewcommand{\algorithmicrequire}{\textbf{\textsc {Entrées  :  } } }
\renewcommand{\algorithmicensure}{\textbf{\textsc { Sortie  :  } } }
\renewcommand{\algorithmicwhile}{\textbf{Tant que}}
\renewcommand{\algorithmicdo}{\textbf{faire }}
\renewcommand{\algorithmicif}{\textbf{Si}}
\renewcommand{\algorithmicelse}{\textbf{Sinon}}
\renewcommand{\algorithmicthen}{\textbf{alors }}
\renewcommand{\algorithmicend}{\textbf{fin}}
\renewcommand{\algorithmicfor}{\textbf{Pour}}
\renewcommand{\algorithmicuntil}{\textbf{Jusqu'à}}
\renewcommand{\algorithmicrepeat}{\textbf{Répéter}}

\newcommand{\A}{\mathbb{A}}
\newcommand{\B}{\mathbb{B}}
\newcommand{\C}{\mathbb{C}}
\newcommand{\D}{\mathbb{D}}
\newcommand{\E}{\mathbb{E}}
\newcommand{\F}{\mathbb{F}}
\newcommand{\G}{\mathbb{G}}
\renewcommand{\H}{\mathbb{H}}
\newcommand{\I}{\mathbb{I}}
\newcommand{\J}{\mathbb{J}}
\newcommand{\K}{\mathbb{K}}
\renewcommand{\L}{\mathbb{L}}
\newcommand{\M}{\mathbb{M}}
\newcommand{\N}{\mathbb{N}}
\renewcommand{\O}{\mathbb{O}}
\renewcommand{\P}{\mathbb{P}}
\newcommand{\Q}{\mathbb{Q}}
\newcommand{\R}{\mathbb{R}}
\renewcommand{\S}{\mathbb{S}}
\newcommand{\T}{\mathbb{T}}
\newcommand{\U}{\mathbb{U}}
\newcommand{\V}{\mathbb{V}}
\newcommand{\W}{\mathbb{W}}
\newcommand{\X}{\mathbb{X}}
\newcommand{\Y}{\mathbb{Y}}
\newcommand{\Z}{\mathbb{Z}}
\newcommand{\Acal}{\mathcal{A}}
\newcommand{\Bcal}{\mathcal{B}}
\newcommand{\Ccal}{\mathcal{C}}
\newcommand{\Dcal}{\mathcal{D}}
\newcommand{\Ecal}{\mathcal{E}}
\newcommand{\Fcal}{\mathcal{F}}
\newcommand{\Gcal}{\mathcal{G}}
\newcommand{\Hcal}{\mathcal{H}}
\newcommand{\Ical}{\mathcal{I}}
\newcommand{\Jcal}{\mathcal{J}}
\newcommand{\Kcal}{\mathcal{K}}
\newcommand{\Lcal}{\mathcal{L}}
\newcommand{\Mcal}{\mathcal{M}}
\newcommand{\Ncal}{\mathcal{N}}
\newcommand{\Ocal}{\mathcal{O}}
\newcommand{\Pcal}{\mathcal{P}}
\newcommand{\Qcal}{\mathcal{Q}}
\newcommand{\Rcal}{\mathcal{R}}
\newcommand{\Scal}{\mathcal{S}}
\newcommand{\Tcal}{\mathcal{T}}
\newcommand{\Ucal}{\mathcal{U}}
\newcommand{\Vcal}{\mathcal{V}}
\newcommand{\Wcal}{\mathcal{W}}
\newcommand{\Xcal}{\mathcal{X}}
\newcommand{\Ycal}{\mathcal{Y}}
\newcommand{\Zcal}{\mathcal{Z}}
\newcommand{\Ascr}{\mathscr{A}}
\newcommand{\Bscr}{\mathscr{B}}
\newcommand{\Cscr}{\mathscr{C}}
\newcommand{\Dscr}{\mathscr{D}}
\newcommand{\Escr}{\mathscr{E}}
\newcommand{\Fscr}{\mathscr{F}}
\newcommand{\Gscr}{\mathscr{G}}
\newcommand{\Hscr}{\mathscr{H}}
\newcommand{\Iscr}{\mathscr{I}}
\newcommand{\Jscr}{\mathscr{J}}
\newcommand{\Kscr}{\mathscr{K}}
\newcommand{\Lscr}{\mathscr{L}}
\newcommand{\Mscr}{\mathscr{M}}
\newcommand{\Nscr}{\mathscr{N}}
\newcommand{\Oscr}{\mathscr{O}}
\newcommand{\Pscr}{\mathscr{P}}
\newcommand{\Qscr}{\mathscr{Q}}
\newcommand{\Rscr}{\mathscr{R}}
\newcommand{\Sscr}{\mathscr{S}}
\newcommand{\Tscr}{\mathscr{T}}
\newcommand{\Uscr}{\mathscr{U}}
\newcommand{\Vscr}{\mathscr{V}}
\newcommand{\Wscr}{\mathscr{W}}
\newcommand{\Xscr}{\mathscr{X}}
\newcommand{\Yscr}{\mathscr{Y}}
\newcommand{\Zscr}{\mathscr{Z}}
\newcommand{\Afrak}{\mathfrak{A}}
\newcommand{\Bfrak}{\mathfrak{B}}
\newcommand{\Cfrak}{\mathfrak{C}}
\newcommand{\Dfrak}{\mathfrak{D}}
\newcommand{\Efrak}{\mathfrak{E}}
\newcommand{\Ffrak}{\mathfrak{F}}
\newcommand{\Gfrak}{\mathfrak{G}}
\newcommand{\Hfrak}{\mathfrak{H}}
\newcommand{\Ifrak}{\mathfrak{I}}
\newcommand{\Jfrak}{\mathfrak{J}}
\newcommand{\Kfrak}{\mathfrak{K}}
\newcommand{\Lfrak}{\mathfrak{L}}
\newcommand{\Mfrak}{\mathfrak{M}}
\newcommand{\Nfrak}{\mathfrak{N}}
\newcommand{\Ofrak}{\mathfrak{O}}
\newcommand{\Pfrak}{\mathfrak{P}}
\newcommand{\Qfrak}{\mathfrak{Q}}
\newcommand{\Rfrak}{\mathfrak{R}}
\newcommand{\Sfrak}{\mathfrak{S}}
\newcommand{\Tfrak}{\mathfrak{T}}
\newcommand{\Ufrak}{\mathfrak{U}}
\newcommand{\Vfrak}{\mathfrak{V}}
\newcommand{\Wfrak}{\mathfrak{W}}
\newcommand{\Xfrak}{\mathfrak{X}}
\newcommand{\Yfrak}{\mathfrak{Y}}
\newcommand{\Zfrak}{\mathfrak{Z}}
\newcommand{\afrak}{\mathfrak{a}}
\newcommand{\bfrak}{\mathfrak{b}}
\newcommand{\cfrak}{\mathfrak{c}}
\newcommand{\dfrak}{\mathfrak{d}}
\newcommand{\efrak}{\mathfrak{e}}
\newcommand{\ffrak}{\mathfrak{f}}
\newcommand{\gfrak}{\mathfrak{g}}
\newcommand{\hfrak}{\mathfrak{h}}
\newcommand{\ifrak}{\mathfrak{i}}
\newcommand{\jfrak}{\mathfrak{j}}
\newcommand{\kfrak}{\mathfrak{k}}
\newcommand{\lfrak}{\mathfrak{l}}
\newcommand{\mfrak}{\mathfrak{m}}
\newcommand{\nfrak}{\mathfrak{n}}
\newcommand{\ofrak}{\mathfrak{o}}
\newcommand{\pfrak}{\mathfrak{p}}
\newcommand{\qfrak}{\mathfrak{q}}
\newcommand{\rfrak}{\mathfrak{r}}
\newcommand{\sfrak}{\mathfrak{s}}
\newcommand{\tfrak}{\mathfrak{t}}
\newcommand{\ufrak}{\mathfrak{u}}
\newcommand{\vfrak}{\mathfrak{v}}
\newcommand{\wfrak}{\mathfrak{w}}
\newcommand{\xfrak}{\mathfrak{x}}
\newcommand{\yfrak}{\mathfrak{y}}
\newcommand{\zfrak}{\mathfrak{z}}
\newcommand{\Abar}{\overline{A}}
\newcommand{\Bbar}{\overline{B}}
\newcommand{\Cbar}{\overline{C}}
\newcommand{\Dbar}{\overline{D}}
\newcommand{\Ebar}{\overline{E}}
\newcommand{\Fbar}{\overline{F}}
\newcommand{\Gbar}{\overline{G}}
\newcommand{\Hbar}{\overline{H}}
\newcommand{\Ibar}{\overline{I}}
\newcommand{\Jbar}{\overline{J}}
\newcommand{\Kbar}{\overline{K}}
\newcommand{\Lbar}{\overline{L}}
\newcommand{\Mbar}{\overline{M}}
\newcommand{\Nbar}{\overline{N}}
\newcommand{\Obar}{\overline{O}}
\newcommand{\Pbar}{\overline{P}}
\newcommand{\Qbar}{\overline{Q}}
\newcommand{\Rbar}{\overline{R}}
\newcommand{\Sbar}{\overline{S}}
\newcommand{\Tbar}{\overline{T}}
\newcommand{\Ubar}{\overline{U}}
\newcommand{\Vbar}{\overline{V}}
\newcommand{\Wbar}{\overline{W}}
\newcommand{\Xbar}{\overline{X}}
\newcommand{\Ybar}{\overline{Y}}
\newcommand{\Zbar}{\overline{Z}}
\newcommand{\abar}{\overline{a}}
\newcommand{\bbar}{\overline{b}}
\newcommand{\cbar}{\overline{c}}
\newcommand{\dbar}{\overline{d}}
\newcommand{\ebar}{\overline{e}}
\newcommand{\fbar}{\overline{f}}
\newcommand{\gbar}{\overline{g}}
\renewcommand{\hbar}{\overline{h}}
\newcommand{\ibar}{\overline{i}}
\newcommand{\jbar}{\overline{j}}
\newcommand{\kbar}{\overline{k}}
\newcommand{\lbar}{\overline{l}}
\newcommand{\mbar}{\overline{m}}
\newcommand{\nbar}{\overline{n}}
\newcommand{\obar}{\overline{o}}
\newcommand{\pbar}{\overline{p}}
\newcommand{\qbar}{\overline{q}}
\newcommand{\rbar}{\overline{r}}
\newcommand{\sbar}{\overline{s}}
\newcommand{\tbar}{\overline{t}}
\newcommand{\ubar}{\overline{u}}
\newcommand{\vbar}{\overline{v}}
\newcommand{\wbar}{\overline{w}}
\newcommand{\xbar}{\overline{x}}
\newcommand{\ybar}{\overline{y}}
\newcommand{\zbar}{\overline{z}}
\newcommand\bigzero{\makebox(0,0){\text{\huge0}}}
\newcommand{\limp}{\lim\limits_{\leftarrow}}
\newcommand{\limi}{\lim\limits_{\rightarrow}}


\DeclareMathOperator{\End}{\mathrm{End}}
\DeclareMathOperator{\Hom}{\mathrm{Hom}}
\DeclareMathOperator{\Vect}{\mathrm{Vect}}
\DeclareMathOperator{\Spec}{\mathrm{Spec}}
\DeclareMathOperator{\multideg}{\mathrm{multideg}}
\DeclareMathOperator{\LM}{\mathrm{LM}}
\DeclareMathOperator{\LT}{\mathrm{LT}}
\DeclareMathOperator{\LC}{\mathrm{LC}}
\DeclareMathOperator{\PPCM}{\mathrm{PPCM}}
\DeclareMathOperator{\PGCD}{\mathrm{PGCD}}
\DeclareMathOperator{\Syl}{\mathrm{Syl}}
\DeclareMathOperator{\Res}{\mathrm{Res}}
\DeclareMathOperator{\Com}{\mathrm{Com}}
\DeclareMathOperator{\GL}{\mathrm{GL}}
\DeclareMathOperator{\SL}{\mathrm{SL}}
\DeclareMathOperator{\SU}{\mathrm{SU}}
\DeclareMathOperator{\SO}{\mathrm{SO}}
\DeclareMathOperator{\Sp}{\mathrm{Sp}}
\DeclareMathOperator{\Spin}{\mathrm{Spin}}
\DeclareMathOperator{\Ker}{\mathrm{Ker}}
%\DeclareMathOperator{\Im}{\mathrm{Im}}

\headheight=0mm
\topmargin=-20mm
\oddsidemargin=-1cm
\evensidemargin=-1cm
\textwidth=18cm
\textheight=25cm
\parindent=0mm
\newif\ifproof
\newcommand{\demo}[1]{\ifproof #1 \else \fi}
 %Instruction d'utilisation : 
%les preuves du texte sont, en principe, entre des balises \demo, en sus des \begin{proof} pour l'instant.
%Laisser le texte tel quel, fait qu'elles ne sont pas affich�es.
%Mettre \prooftrue fait que toutes les preuves jusqu'� un \prooffalse ou la fin du document. 


 \begin{document}
\newtheorem{Thm}{Th�or�me}[section]
\newtheorem{Prop}[Thm]{Proposition}
\newtheorem{Propte}[Thm]{Propri�t�}
\newtheorem{Lemme}[Thm]{Lemme}
\newtheorem{Cor}[Thm]{Corollaire}


\theoremstyle{definition}

\newtheorem{Ex}[Thm]{Exemple}
\newtheorem{Def}[Thm]{D�finition}
\newtheorem{Defpropte}[Thm]{D�finition et propri�t�}
\newtheorem{Defprop}[Thm]{D�finition et proposition}
\newtheorem{Defthm}[Thm]{Th�or�me et d�finition}
\newtheorem{Not}[Thm]{Notation}
\newtheorem{Conv}[Thm]{Convention}
\newtheorem{Cons}[Thm]{Construction}

\theoremstyle{remark}
\newtheorem{Rq}[Thm]{Remarque}
\newtheorem{Slog}[Thm]{Slogan}
\newtheorem{Exo}[Thm]{Exercice}
\fi

\section{Alg�bre ext�rieure}
Soit $E$ un espace vectoriel de dimension finie sur un corps commutatif $k$. \\
On appelle $p$�me puissance ext�rieure de $E$ l'espace vectoriel $\bigwedge^p E$,quotient de $E^{\otimes p}$ par le sous-espace vectoriel $\Zcal_p=\Vect(x_{1} \otimes \ldots \otimes x_{p} | x_i \in E,\exists m \neq n, x_{m}=x_{n})$ (par convention, $\bigwedge^0 E=k$). \\
On note $x_1 \wedge \ldots \wedge x_p$ la classe d'�quivalence $x_1 \ldots \ldots \ldots x_p$. \\
Le passage au quotient conserve les propri�t�s d'associativit�e et de $p$-lin�arit� et fait que $\forall u,v \in E, u \wedge v=-v \times u$ \\
On appelle alg�bre ext�rieure la somme directe des puissances ext�rieures : $\bigwedge^* E=\bigoplus_{p \in \N} \bigwedge^p E$. On munit $\bigwedge^* E$ d'une structure d'alg�bre gradu�e gr�ce au produit : $(x_1 \wedge \ldots x_p,y_1 \wedge \ldots y_q) \in \bigwedge^p E \times \bigwedge^q E \mapsto x_1 \wedge \ldots x_p \wedge y_1 \wedge \ldots y_q \in \bigwedge^{p+q} E$.


\begin{Lemme}
$\Zcal_p=\Vect(x_{1} \otimes \ldots \otimes x_{p} | x_i \in E, \{x_1,\ldots,x_p\} \text{ est une famille li�e })$
\end{Lemme}
\begin{proof}
Posons $V=\Vect(x_{1} \otimes \ldots \otimes x_{p} | x_i \in E, \{x_1,\ldots,x_p\} \text{ est une famille li�e })$. \\
On a clairement $\Zcal_p \subset V$. R�ciproquement, soit $x_1 \wedge \ldots \wedge x_p \in V$ i.e.  $\{x_1,\ldots,x_p\}$ est li�e. On a donc un $i\in [\![1,n]\!]$ et des $\lambda_j \in k,j \neq i$ tel que $x_i=\sum_{j \neq i} \lambda_jx_j$ et donc : \\
 $x_1 \wedge \ldots \wedge x_i \wedge \ldots \wedge x_p=\sum_{j \neq i} \lambda_j x_1 \wedge \ldots \wedge x_j \wedge \ldots \wedge x_p \in \Zcal_p$ 
\end{proof}
Cela nous permet de voir que $\bigwedge^p E= \{0\}$ si $p>dim(E)$

\begin{Prop}
Soit $\Bcal=\{e_1,\ldots,e_n\}$ une base de $E$. Pour $p \leq n$,$\{e_{i_1} \wedge \ldots \wedge e_{i_p},1 \leq i_1<\ldots <i_p \leq n\}$ est une base de $\bigwedge^p E$.
\end{Prop}
\begin{proof}
Comme la famille $\{e_1 \otimes \ldots \otimes e_p|e_i \in \Bcal\}$ est une base de $E^{\otimes p}$ alors le passage au quotient fait que l'espace quotient a comme base $ \{e_1 \wedge \ldots \wedge e_p|e_i \in \Bcal, i \neq j \Rightarrow e_i \neq e_j\}$. En ordonnant les $e_i$; on obtient le r�sultat. 
\end{proof}

\begin{Cor}
$\dim_k \bigwedge^p E=\binom{\dim E}{p}$
\end{Cor}
En particulier, $\dim_k \bigwedge^{\dim E} E=1$ (cela permet de d�finir le d�terminant)

\begin{Cor}
Soient $\{e_1,\ldots,e_m\},\{f_1,\ldots,f_m\}$ deux familles libres d'�l�ments de $E$. Alors $\Vect(e_1,\ldots,e_m)=\Vect(f_1,\ldots,f_m)$ si, et seulement si, $e_1 \wedge \ldots \wedge e_m$ et $f_1 \wedge \ldots \wedge f_m$ sont colin�aires.
\end{Cor}

\begin{proof}
$\Rightarrow$ : Soit $W=\Vect(e_1,\ldots,e_m)$. Alors $e_1 \wedge \ldots \wedge e_m$, $f_1 \wedge \ldots \wedge f_m \in \bigwedge^k W$. Ce dernier ensemble �tant de dimension 1 et donc $e_1 \wedge \ldots \wedge e_m$ et $f_1 \wedge \ldots \wedge f_m$ sont colin�aires.\\
$\Leftarrow$ : Soit $W=\Vect(e_1,\ldots,e_m)$ et $W'=\Vect(f_1,\ldots,f_m)$. Si $f_i \notin W$ alors on compl�te $\{f_i,e_1,\ldots,e_m\}$ en une base $\{v_1,\ldots,v_n\}$ de $E$.
Pour donner les id�es, prenons $i=1$, \\
$\lambda e_1 \wedge \ldots \wedge e_m=f_1 \wedge f_2 \wedge \ldots \wedge f_n=f_1 \wedge \sum_{j_2} a_{2j_2} v_{j_2} \wedge \ldots \wedge \sum_{j_m} a_{mj_m} v_{j_m}=\sum_{j_2}\cdots\sum_{j_m}f_1 \wedge  a_{2j_2} v_{j_2} \wedge \ldots \wedge  a_{mj_m} v_{j_m}$.
A droite de l'�galit�, $f_1$ apparait partout alors qu'� droite, il n'intervient pas. Ce qui est absurde car $\lambda \neq 0$.
\end{proof}

\section{Plongement de Pl�cker}
\begin{Def}[Plongement de Pl�cker]
On appelle plongement de Pl�cker l'application $\psi_{r,n} : G(r,n) \to P(\bigwedge^r k^n)$ qui envoie un sous-espace vectoriel $W$ de dimension $r$ de $k^n$ de base $e_1,\ldots,e_r$ sur $[e_1 \wedge \ldots \wedge e_r]$
\end{Def}
\begin{Cor}
$\psi_{r,n}$ ne d�pend pas du choix de la base et est injectif.
\end{Cor}

Nous allons maintenant d�terminer des �quations polynomiales v�rifi�es par $\psi_{r,n}(G(r,n))$.

\begin{Lemme}
Soit $W \in G(r,n)$ et $\{e_1,\ldots,e_r\}$ une base de $W$. Alors $W=\{x \in k^n |x \wedge (e_1 \wedge \ldots \wedge e_r)=0\}$ 
\end{Lemme}
\begin{proof}
Posons $E_W=\{x \in k^n |x \wedge (e_1 \wedge \ldots \wedge e_r)=0\}$.
On remarque ais�ment que $W$ est un espace vectoriel et que $W \subset E_W$. \\
Soit $x \in E_W$ ie $x \wedge (e_1 \wedge \ldots \wedge e_r)=0$ et donc $\{x,e_1,\ldots,e_r\}$ est une famille li�e. Comme $\{e_1,\ldots,e_r\}$ est une base alors $x \in \Vect(e_1,\ldots,e_r)=W$
\end{proof}
\begin{Lemme}
Soit $\Lambda \in \bigwedge^r k^n$ tel que $W=\{ x \in k^n | \Lambda \wedge x=0\}$ soit de dimension $r$. Alors $\psi_{r,n}(W)=[\Lambda]$
\end{Lemme}
\begin{proof}
%Pour montrer ce lemme, on peut tout d'abord remarquer que l'action de $\GL_n(k)$ commute avec $\psi_{r,n}$. $\GL_n(k)$ agit sur $P(\bigwedge^r k^n$ par $P \cdot [e_1 \wedge \ldots \wedge e_r]=[Pe_1 \wedge \ldots \wedge Pe_r]$. De plus, comme $\{Pe_1,\ldots, Pe_r\}$ est une base de $W$ alors $\psi_{r,n}(PW)=P\cdot \psi_{r,n}(W)$. \\
Soit $\{e_1,\ldots,e_r\}$ une base de $W$. On peut compl�ter cette base en un base $\{e_1,\ldots,e_n\}$ de $k^n$. \\
On peut d�composer $\Lambda$ sur la base associ�e de $\bigwedge^r k^n$ : \\
$\Lambda=\sum_p \lambda_p e_{i_{1p}} \wedge \ldots \wedge e_{i_{rp}}$. \\
Comme $\Lambda \wedge e_i=0$ alors $\Lambda=\sum_{p,i_{jp} \neq 0} \lambda_p e_1 \wedge e_{i_{1p}} \wedge \ldots \wedge e_{i_{rp}}$. \\
Comme les $e_1 \wedge e_{i_{1p}} \wedge \ldots \wedge e_{i_{rp}}$ forment une famille libre alors les $\lambda_p$ correspondant sont nulles.
En it�rant ce processus, on trouve que les scalaires correspondant aux multi-vecteurs contenant $e_j$ pour $j>r$ sont nuls et donc $\Lambda=\lambda e_1 \wedge \ldots \wedge e_r$, ce qu'il fallait d�montrer.
\end{proof}






\ifwhole
 \end{document}
\fi