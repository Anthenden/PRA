
\newif\ifwhole

\wholetrue
% Ajouter \wholetrue si on compile seulement ce fichier

\ifwhole
 \documentclass[a4page,10pt]{article}
     \usepackage[Latin1]{inputenc}
\usepackage[francais]{babel}
\usepackage{amsmath,amssymb,amsthm}
\usepackage{textcomp}
\usepackage{mathrsfs}
\usepackage{algcompatible,algorithm  }
\usepackage[all]{xy}
\usepackage{hyperref}
\usepackage{fancyhdr}
\usepackage{supertabular}
\pagestyle{plain}

\toks0=\expandafter{\xy}
\edef\xy{\noexpand\shorthandoff{!?;:}\the\toks0 }
\makeatletter

\renewcommand*{\ALG@name}{Algorithme}

\makeatother

\renewcommand{\algorithmicrequire}{\textbf{\textsc {Entrées  :  } } }
\renewcommand{\algorithmicensure}{\textbf{\textsc { Sortie  :  } } }
\renewcommand{\algorithmicwhile}{\textbf{Tant que}}
\renewcommand{\algorithmicdo}{\textbf{faire }}
\renewcommand{\algorithmicif}{\textbf{Si}}
\renewcommand{\algorithmicelse}{\textbf{Sinon}}
\renewcommand{\algorithmicthen}{\textbf{alors }}
\renewcommand{\algorithmicend}{\textbf{fin}}
\renewcommand{\algorithmicfor}{\textbf{Pour}}
\renewcommand{\algorithmicuntil}{\textbf{Jusqu'à}}
\renewcommand{\algorithmicrepeat}{\textbf{Répéter}}

\newcommand{\A}{\mathbb{A}}
\newcommand{\B}{\mathbb{B}}
\newcommand{\C}{\mathbb{C}}
\newcommand{\D}{\mathbb{D}}
\newcommand{\E}{\mathbb{E}}
\newcommand{\F}{\mathbb{F}}
\newcommand{\G}{\mathbb{G}}
\renewcommand{\H}{\mathbb{H}}
\newcommand{\I}{\mathbb{I}}
\newcommand{\J}{\mathbb{J}}
\newcommand{\K}{\mathbb{K}}
\renewcommand{\L}{\mathbb{L}}
\newcommand{\M}{\mathbb{M}}
\newcommand{\N}{\mathbb{N}}
\renewcommand{\O}{\mathbb{O}}
\renewcommand{\P}{\mathbb{P}}
\newcommand{\Q}{\mathbb{Q}}
\newcommand{\R}{\mathbb{R}}
\renewcommand{\S}{\mathbb{S}}
\newcommand{\T}{\mathbb{T}}
\newcommand{\U}{\mathbb{U}}
\newcommand{\V}{\mathbb{V}}
\newcommand{\W}{\mathbb{W}}
\newcommand{\X}{\mathbb{X}}
\newcommand{\Y}{\mathbb{Y}}
\newcommand{\Z}{\mathbb{Z}}
\newcommand{\Acal}{\mathcal{A}}
\newcommand{\Bcal}{\mathcal{B}}
\newcommand{\Ccal}{\mathcal{C}}
\newcommand{\Dcal}{\mathcal{D}}
\newcommand{\Ecal}{\mathcal{E}}
\newcommand{\Fcal}{\mathcal{F}}
\newcommand{\Gcal}{\mathcal{G}}
\newcommand{\Hcal}{\mathcal{H}}
\newcommand{\Ical}{\mathcal{I}}
\newcommand{\Jcal}{\mathcal{J}}
\newcommand{\Kcal}{\mathcal{K}}
\newcommand{\Lcal}{\mathcal{L}}
\newcommand{\Mcal}{\mathcal{M}}
\newcommand{\Ncal}{\mathcal{N}}
\newcommand{\Ocal}{\mathcal{O}}
\newcommand{\Pcal}{\mathcal{P}}
\newcommand{\Qcal}{\mathcal{Q}}
\newcommand{\Rcal}{\mathcal{R}}
\newcommand{\Scal}{\mathcal{S}}
\newcommand{\Tcal}{\mathcal{T}}
\newcommand{\Ucal}{\mathcal{U}}
\newcommand{\Vcal}{\mathcal{V}}
\newcommand{\Wcal}{\mathcal{W}}
\newcommand{\Xcal}{\mathcal{X}}
\newcommand{\Ycal}{\mathcal{Y}}
\newcommand{\Zcal}{\mathcal{Z}}
\newcommand{\Ascr}{\mathscr{A}}
\newcommand{\Bscr}{\mathscr{B}}
\newcommand{\Cscr}{\mathscr{C}}
\newcommand{\Dscr}{\mathscr{D}}
\newcommand{\Escr}{\mathscr{E}}
\newcommand{\Fscr}{\mathscr{F}}
\newcommand{\Gscr}{\mathscr{G}}
\newcommand{\Hscr}{\mathscr{H}}
\newcommand{\Iscr}{\mathscr{I}}
\newcommand{\Jscr}{\mathscr{J}}
\newcommand{\Kscr}{\mathscr{K}}
\newcommand{\Lscr}{\mathscr{L}}
\newcommand{\Mscr}{\mathscr{M}}
\newcommand{\Nscr}{\mathscr{N}}
\newcommand{\Oscr}{\mathscr{O}}
\newcommand{\Pscr}{\mathscr{P}}
\newcommand{\Qscr}{\mathscr{Q}}
\newcommand{\Rscr}{\mathscr{R}}
\newcommand{\Sscr}{\mathscr{S}}
\newcommand{\Tscr}{\mathscr{T}}
\newcommand{\Uscr}{\mathscr{U}}
\newcommand{\Vscr}{\mathscr{V}}
\newcommand{\Wscr}{\mathscr{W}}
\newcommand{\Xscr}{\mathscr{X}}
\newcommand{\Yscr}{\mathscr{Y}}
\newcommand{\Zscr}{\mathscr{Z}}
\newcommand{\Afrak}{\mathfrak{A}}
\newcommand{\Bfrak}{\mathfrak{B}}
\newcommand{\Cfrak}{\mathfrak{C}}
\newcommand{\Dfrak}{\mathfrak{D}}
\newcommand{\Efrak}{\mathfrak{E}}
\newcommand{\Ffrak}{\mathfrak{F}}
\newcommand{\Gfrak}{\mathfrak{G}}
\newcommand{\Hfrak}{\mathfrak{H}}
\newcommand{\Ifrak}{\mathfrak{I}}
\newcommand{\Jfrak}{\mathfrak{J}}
\newcommand{\Kfrak}{\mathfrak{K}}
\newcommand{\Lfrak}{\mathfrak{L}}
\newcommand{\Mfrak}{\mathfrak{M}}
\newcommand{\Nfrak}{\mathfrak{N}}
\newcommand{\Ofrak}{\mathfrak{O}}
\newcommand{\Pfrak}{\mathfrak{P}}
\newcommand{\Qfrak}{\mathfrak{Q}}
\newcommand{\Rfrak}{\mathfrak{R}}
\newcommand{\Sfrak}{\mathfrak{S}}
\newcommand{\Tfrak}{\mathfrak{T}}
\newcommand{\Ufrak}{\mathfrak{U}}
\newcommand{\Vfrak}{\mathfrak{V}}
\newcommand{\Wfrak}{\mathfrak{W}}
\newcommand{\Xfrak}{\mathfrak{X}}
\newcommand{\Yfrak}{\mathfrak{Y}}
\newcommand{\Zfrak}{\mathfrak{Z}}
\newcommand{\afrak}{\mathfrak{a}}
\newcommand{\bfrak}{\mathfrak{b}}
\newcommand{\cfrak}{\mathfrak{c}}
\newcommand{\dfrak}{\mathfrak{d}}
\newcommand{\efrak}{\mathfrak{e}}
\newcommand{\ffrak}{\mathfrak{f}}
\newcommand{\gfrak}{\mathfrak{g}}
\newcommand{\hfrak}{\mathfrak{h}}
\newcommand{\ifrak}{\mathfrak{i}}
\newcommand{\jfrak}{\mathfrak{j}}
\newcommand{\kfrak}{\mathfrak{k}}
\newcommand{\lfrak}{\mathfrak{l}}
\newcommand{\mfrak}{\mathfrak{m}}
\newcommand{\nfrak}{\mathfrak{n}}
\newcommand{\ofrak}{\mathfrak{o}}
\newcommand{\pfrak}{\mathfrak{p}}
\newcommand{\qfrak}{\mathfrak{q}}
\newcommand{\rfrak}{\mathfrak{r}}
\newcommand{\sfrak}{\mathfrak{s}}
\newcommand{\tfrak}{\mathfrak{t}}
\newcommand{\ufrak}{\mathfrak{u}}
\newcommand{\vfrak}{\mathfrak{v}}
\newcommand{\wfrak}{\mathfrak{w}}
\newcommand{\xfrak}{\mathfrak{x}}
\newcommand{\yfrak}{\mathfrak{y}}
\newcommand{\zfrak}{\mathfrak{z}}
\newcommand{\Abar}{\overline{A}}
\newcommand{\Bbar}{\overline{B}}
\newcommand{\Cbar}{\overline{C}}
\newcommand{\Dbar}{\overline{D}}
\newcommand{\Ebar}{\overline{E}}
\newcommand{\Fbar}{\overline{F}}
\newcommand{\Gbar}{\overline{G}}
\newcommand{\Hbar}{\overline{H}}
\newcommand{\Ibar}{\overline{I}}
\newcommand{\Jbar}{\overline{J}}
\newcommand{\Kbar}{\overline{K}}
\newcommand{\Lbar}{\overline{L}}
\newcommand{\Mbar}{\overline{M}}
\newcommand{\Nbar}{\overline{N}}
\newcommand{\Obar}{\overline{O}}
\newcommand{\Pbar}{\overline{P}}
\newcommand{\Qbar}{\overline{Q}}
\newcommand{\Rbar}{\overline{R}}
\newcommand{\Sbar}{\overline{S}}
\newcommand{\Tbar}{\overline{T}}
\newcommand{\Ubar}{\overline{U}}
\newcommand{\Vbar}{\overline{V}}
\newcommand{\Wbar}{\overline{W}}
\newcommand{\Xbar}{\overline{X}}
\newcommand{\Ybar}{\overline{Y}}
\newcommand{\Zbar}{\overline{Z}}
\newcommand{\abar}{\overline{a}}
\newcommand{\bbar}{\overline{b}}
\newcommand{\cbar}{\overline{c}}
\newcommand{\dbar}{\overline{d}}
\newcommand{\ebar}{\overline{e}}
\newcommand{\fbar}{\overline{f}}
\newcommand{\gbar}{\overline{g}}
\renewcommand{\hbar}{\overline{h}}
\newcommand{\ibar}{\overline{i}}
\newcommand{\jbar}{\overline{j}}
\newcommand{\kbar}{\overline{k}}
\newcommand{\lbar}{\overline{l}}
\newcommand{\mbar}{\overline{m}}
\newcommand{\nbar}{\overline{n}}
\newcommand{\obar}{\overline{o}}
\newcommand{\pbar}{\overline{p}}
\newcommand{\qbar}{\overline{q}}
\newcommand{\rbar}{\overline{r}}
\newcommand{\sbar}{\overline{s}}
\newcommand{\tbar}{\overline{t}}
\newcommand{\ubar}{\overline{u}}
\newcommand{\vbar}{\overline{v}}
\newcommand{\wbar}{\overline{w}}
\newcommand{\xbar}{\overline{x}}
\newcommand{\ybar}{\overline{y}}
\newcommand{\zbar}{\overline{z}}
\newcommand\bigzero{\makebox(0,0){\text{\huge0}}}
\newcommand{\limp}{\lim\limits_{\leftarrow}}
\newcommand{\limi}{\lim\limits_{\rightarrow}}


\DeclareMathOperator{\End}{\mathrm{End}}
\DeclareMathOperator{\Hom}{\mathrm{Hom}}
\DeclareMathOperator{\Vect}{\mathrm{Vect}}
\DeclareMathOperator{\Spec}{\mathrm{Spec}}
\DeclareMathOperator{\multideg}{\mathrm{multideg}}
\DeclareMathOperator{\LM}{\mathrm{LM}}
\DeclareMathOperator{\LT}{\mathrm{LT}}
\DeclareMathOperator{\LC}{\mathrm{LC}}
\DeclareMathOperator{\PPCM}{\mathrm{PPCM}}
\DeclareMathOperator{\PGCD}{\mathrm{PGCD}}
\DeclareMathOperator{\Syl}{\mathrm{Syl}}
\DeclareMathOperator{\Res}{\mathrm{Res}}
\DeclareMathOperator{\Com}{\mathrm{Com}}
\DeclareMathOperator{\GL}{\mathrm{GL}}
\DeclareMathOperator{\SL}{\mathrm{SL}}
\DeclareMathOperator{\SU}{\mathrm{SU}}
\DeclareMathOperator{\SO}{\mathrm{SO}}
\DeclareMathOperator{\Sp}{\mathrm{Sp}}
\DeclareMathOperator{\Spin}{\mathrm{Spin}}
\DeclareMathOperator{\Ker}{\mathrm{Ker}}
%\DeclareMathOperator{\Im}{\mathrm{Im}}

\headheight=0mm
\topmargin=-20mm
\oddsidemargin=-1cm
\evensidemargin=-1cm
\textwidth=18cm
\textheight=25cm
\parindent=0mm
\newif\ifproof
\newcommand{\demo}[1]{\ifproof #1 \else \fi}
 %Instruction d'utilisation : 
%les preuves du texte sont, en principe, entre des balises \demo, en sus des \begin{proof} pour l'instant.
%Laisser le texte tel quel, fait qu'elles ne sont pas affich�es.
%Mettre \prooftrue fait que toutes les preuves jusqu'� un \prooffalse ou la fin du document. 


 \begin{document}
\newtheorem{Thm}{Th�or�me}[section]
\newtheorem{Prop}[Thm]{Proposition}
\newtheorem{Propte}[Thm]{Propri�t�}
\newtheorem{Lemme}[Thm]{Lemme}
\newtheorem{Cor}[Thm]{Corollaire}


\theoremstyle{definition}

\newtheorem{Ex}[Thm]{Exemple}
\newtheorem{Def}[Thm]{D�finition}
\newtheorem{Defpropte}[Thm]{D�finition et propri�t�}
\newtheorem{Defprop}[Thm]{D�finition et proposition}
\newtheorem{Defthm}[Thm]{Th�or�me et d�finition}
\newtheorem{Not}[Thm]{Notation}
\newtheorem{Conv}[Thm]{Convention}
\newtheorem{Cons}[Thm]{Construction}

\theoremstyle{remark}
\newtheorem{Rq}[Thm]{Remarque}
\newtheorem{Slog}[Thm]{Slogan}
\newtheorem{Exo}[Thm]{Exercice}
\fi



\begin{Lemme}\label{lem_calcul}
Soit $T=(f,g,h)$ un triplet de Darboux avec $f$ une immersion. Alors,
	\begin{equation}\label{eq}
	\theta_+ \circ (\varphi_T)_+=(h:-h \cdot \widetilde{g} : \widetilde{g} : 1)=c \circ \varphi_T
\end{equation}
o� $c : \Ocal \to \Ocal$ est l'application d�finie par : $c(x:s:y:t)=(x : t: y :s)$
\end{Lemme}

\begin{proof}
$c$ est la transformation projective induite par $\rho$ par la conjugaison de $\Zcal$ : 
\begin{eqnarray*}
c(x:s:y:t)&=&\rho^{-1} \circ \overline{\;\cdot\;} \circ \rho(x :s :y :t)=\rho^{-1}\overline{\begin{pmatrix} s & x \\ y &  -t\end{pmatrix}} \\
          &=&\rho^{-1}\begin{pmatrix} -t & -x \\ -y & s\end{pmatrix}=(x :t: y : s)
\end{eqnarray*}
On sait,d'apr�s la calcul de $(\varphi_T)_+(=(h,\widetilde{g})_+)$ dans le th�or�me \ref{} et l'identification par $\rho$ de $(a,b)_+$ avec $\Ker(L_Z)$ o� $Z=\begin{pmatrix} a \cdot b& -a \\ -b & 1\end{pmatrix}$ vu dans le th�or�me \ref{}, que :
	\[\rho_+((\varphi_T)_+)=\begin{pmatrix} h \cdot \widetilde{g} & -h \\ -\widetilde{g} & 1\end{pmatrix}
\]
On peut remarquer que :  
	\[\begin{pmatrix} h \cdot \widetilde{g} & -h \\ -\widetilde{g} & 1\end{pmatrix}=\overline{\begin{pmatrix}  1& h \\ \widetilde{g} & h \cdot \widetilde{g}\end{pmatrix}}=\rho(h:1: \widetilde{g}:h \cdot \widetilde{g})=\rho(c \circ \varphi_{A(T)})
\]
On obtient le r�sultat voulu en composant avec $\rho^{-1}$ � gauche.

\end{proof}

\begin{Prop}\label{transformation}
Il existe des applications $a$ et $d$ de l'ensemble des immersions totalement isotropes $\varphi : S \to Q$  vers les applications totalement isotropes de $S$ dans $Q$, telles que si $T=(f,g,h)$ est un triplet de Darboux pour lequel $\varphi_T$ est une immersion (i.e. $f$ est une immersion), $a(\varphi_T)=\varphi_{A(T)}$ et que si, de plus, $D(T)$ est d�fini, $d(\varphi_T)=\varphi(D(T))$
\end{Prop}

\begin{proof}
On posant $a=c \circ \theta_+ \circ \varphi_+$ et $d : (x : s :y :t) \mapsto (y : s : x :t)$
\end{proof}

Dans la suite, on identifiera $Q$ et $Q_\pm$ et sous-entendra les applications $\theta_\pm$.

\begin{Prop}\label{composition}
Si $\varphi : S \to Q$ est une immersion totalement isotrope telle que $\varphi_+$ soit une immersion, on a $\varphi_{++}=\varphi_-$ et $\varphi_{+-}=\varphi$. De m�me, si $\varphi$ est une immersion, on a $\varphi_{--}=\varphi_+$ et $\varphi_{-+}=\varphi$
\end{Prop}

\begin{proof}
D'apr�s la trialit� diff�rentielle du th�or�me \ref{}, $\psi_+,d\psi_+ \in \Ker(L_{\psi_-}) \cap \Ker(R_{\psi}) $. Ce dernier ensemble est un espace vectoriel totalement isotrope de dimension 3 car $\psi\psi_-=0$. On en conclut que $(\varphi_+)_+=\varphi_-$ et $(\varphi_+)_-=\varphi$. En effet, $V:=\Ker(L_{\psi_-}) \cap \Ker(R_{\psi}) $ est inclus dans $\Ker(L_{\psi_-})$ et $\Ker(R_{\psi}) $ ; ces deux espaces sont les deux SETIM contenant $V$. $\Ker(L_{\psi_-})$ est le SETIM correspondant � un rotation de $\SO(4)$ et $\Ker(R_{\psi})$ est le SETIM correspondant � un rotation de $O^-(4)$. \\
Le m�me raisonnement fonctionne pour $\varphi_-$.
\end{proof}

De la m�me, on montre que si $\sigma$ est un anti-automorphisme de $\Zcal$ (c'est-�-dire $\forall x,y \in \Zcal, \sigma(xy)=\sigma(y)\sigma(x)$ alors $(\sigma \circ \varphi)_+=c \circ \varphi_-$ et si c'est un automorphisme, $(\sigma \circ \varphi)_+=c \circ \varphi_+$. Maintenant, on peut montrer le

\begin{Thm}[Darboux]
La transformation $(D \circ A)^6$ est l'identit� sur l'ensemble des immersions totalement isotropes $\varphi : S \to Q$ telles que $\varphi_{\pm}$ soient aussi des immersions
\end{Thm}

\begin{proof}
Soit $T=(f,g,h)$ un triplet de Darboux telle que $(\varphi_T)_{\pm}$ soient aussi des immersions. \\
On notera $\varphi_T$ par $\varphi$. \\
En utilisant les propositions \ref{transformation} et \ref{composition} et la remarque que l'on a faite avant l'�nonc� du th�or�me,on obtient :
	\[d \circ a(\varphi)=\sigma \circ c \circ \varphi_+ ; (d \circ a)^2(\varphi)=\varphi_- ; (d \circ a)^3(\varphi)=\sigma \circ c \circ \varphi
\]
\[(d \circ a)^4(\varphi)= \varphi_+ ; (d \circ a)^5(\varphi)=\sigma \circ c \circ \varphi_- ; (d \circ a)^6(\varphi)=\varphi
\]
\end{proof}
Ainsi le groupe engendr� par $a$ et $d$ est le groupe di�dral � 12 �l�ments. Les six autres �l�ments sont :

\[a(\varphi)=c \circ \varphi_+  ; a \circ d \circ a(\varphi)=\sigma \circ \varphi_- ; a \circ (d \circ a)^2(\varphi)=c \circ \varphi ; 
\]
\[a \circ (d \circ a)^3(\varphi)=\varphi_+ ; a \circ (d \circ a)^4(\varphi)= c \circ \varphi_- ; d=a \circ (d \circ a)^5(\varphi)=\sigma \circ \varphi ; 
\]


\ifwhole
 \end{document}
\fi


















\begin{proof}

\end{proof}