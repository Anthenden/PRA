
\newif\ifwhole

\wholetrue
% Ajouter \wholetrue si on compile seulement ce fichier

\ifwhole
 \documentclass[a4page,10pt]{article}
     \input{entete}
 \begin{document}
\input{enonce}

\fi
%$\H$ est une $\R$-alg�bre non commutative ($ij \neq ji$). \\
%$\H$ s'identifie � $\R^4$ gr�ce au choix de la base $(1,i,j,k)$. On munit $\H$ de la norme euclienne i.e. $|a+bi+cj+dk|=\sqrt{a^2+b^2+c^2+d^2}$ 
%\begin{Prop}
%Le centre $Z(\H)$ de $\H$ est $\R$
%\end{Prop}
%\begin{proof}
%On a clairement $\R \subset Z(\H)$. Montrons l'inclusion r�ciproque : \\
%Soit $x=a+bi+cj+dk \in Z(\H)$. Alors $\forall u \in \H, xu=ux$.\\
%En particulier, pour $u=i$, $ai-b-ck+dj=ai-b+ck-dj$ et donc $c=d=0$, et pour $u=j$, $aj-bk=aj+bk$ i.e. $b=0$. Cela montre que $x=a \in \R$
%\end{proof}
%\begin{Def}[Conjugaison]
%Le conjugu� du quaternion $a+bi+cj+dk$ est $\overline{a+bi+cj+dk}=a-bi-cj-dk$
%\end{Def}
%\begin{Propte} 
%\begin{enumerate}  
	%\item la conjugaison est involutive
	%\item $\forall x,y \in \H,\overline{xy}=\overline{y}\;\overline{x}$
%\end{enumerate}
%\end{Propte}
%\begin{proof}
%\begin{enumerate}
	%\item Evident
	%\item Calculs
%\end{enumerate}
%\end{proof}
%
%
%\begin{Propte}[de la norme sur les quaternions] 
%\begin{enumerate}
	%\item $\forall q \in \H, |q|^2=q\overline{q}$ 
	%\item $\forall x,y \in \H,|xy|=|x||y|$
	%\item $|\cdot |$ est un morphisme de groupes $(\H^*, \cdot) \to (\R^*_+,\cdot)$
%\end{enumerate}
%\end{Propte}
%\begin{proof}
%\begin{enumerate} 
	%\item On d�veloppe le produit.
	%\item Soit $x,y \in \H$. 
	%$|xy|=xy\overline{xy}=xy\overline{y}\;\overline{x}=|x||y|$
	%\item D�coule du point pr�c�dent
%\end{enumerate}
%
%\end{proof}
\begin{Thm}
$SO(4) \simeq S^3 \times S^3/\{\pm(1,1)\}$ (en tant que groupes de Lie)  
\end{Thm}
\begin{proof}
Soit $\varphi : (u,v) \in S^3 \times S^3 \mapsto (x \mapsto uxv^{-1}=ux\overline{v}) \in \GL(\H)$  o� on a identifi� $\R^4$ et $\H$. \\
$\varphi$ est clairement un morphisme de groupes de classe $\Ccal^\infty$ (et bien � valeurs dans $\GL(\H)$ car $\varphi(u,v)$ est bijective d'inverse $\varphi(u^{-1},v^{-1})$)
D�terminons maintenant son noyau et son image. \\
\textbf{Noyau de $\varphi$ :} \\
Soit $(u,v) \in \Ker(\varphi)$. Alors, pour tout $x \in \H$, $uxv^{-1}=x \;(*)$.\\ 
En particulier, en prenant $x=1$, on a $u=v$. Maintenant, $(*)$ nous dit que $u \in Z(\H)=\R$. Comme $u \in S^3$, on d�duit que $u=\pm 1$. \\
R�ciproquement, $\pm (1,1) \in \Ker(\varphi)$. \\
On a donc : $\Ker(\varphi)=\{\pm(1,1)\}$. \\
\textbf{Image de $\varphi$ :} \\
Soient $(u,v) \in S^3 \times S^3$ et $x \in \H$.\\
$|\varphi(u,v)(x)|=|uxv^{-1}|=|u||x||v^{-1}|=|x|$. Ce qui montre que $\varphi(u,v) \in O(4)$ et donc $Im(\varphi) \subset O(4)$.\\
On peut affiner cela en disant que, comme $S^3 \times S^3$ est connexe, $Id \in Im(\varphi)$ et $\varphi$ est continu alors $Im(\varphi)$ est inclus dans la composante connexe de $O(4)$ contenant $Id$ i.e. $SO(4)$. On peut donc consid�rer que l'ensemble d'arriv�e de $\varphi$ est $SO(4)$. \\
Montrons maintenant que $\varphi$ est un diff�omorphisme local. Pour cela, il suffit de le montrer en $(1,1)$ (les translations sont des diff�omorphismes) et gr�ce au th�or�me d'inversion locale, il suffit de montrer que $d_{(1,1)} \varphi : T_{(1,1)} S^3 \times S^3 \to \sfrak\ofrak(4)$ est un isomorphisme. \\
$\varphi$ �tant bilin�aire, on sait que, pour tout $(\delta,\varepsilon) \in T_{(1,1)} S^3 \times S^3=\{ x+y+z+t=0,x'+y'+z'+t'=0\}$,
$\forall x \in \H, d_{(1,1)} \varphi(\delta,\varepsilon)(x)=\delta x+x\overline{\varepsilon}$. \\
Soit $(\delta,\varepsilon) \in \Ker(d_{(1,1)} \varphi)$. Alors $\forall x\in \H, \delta x=-x\overline{\varepsilon}$. \\
En particulier, pour $x=1$, on obtient $\delta=-\overline{\varepsilon}$. On a donc $\forall x\in \H,\delta x=x \delta$ i.e. $\delta \in \R$. Comme, de plus, $\delta \in \{x+y+z+t=0\}$ alors $\delta=0$. Cela montre que $\Ker(d_{(1,1)}\varphi)=\{0\}$ et comme $\dim(T_{1,1} S^3 \times S^3)=\dim(\sfrak\ofrak(4))=6$ alors $d_{(1,1)}\varphi$ est un isomorphisme. \\
Tout cela nous permet de dire que $Im(\varphi)$ est un ouvert de $SO(4)$. \\
$S^3 \times S^3$ est compact donc $Im(\varphi)=\varphi(S^3 \times S^3)$ est compact et est donc ferm�. \\
Par connexit� de $\SO(4)$, $Im(\varphi)=\SO(4)$.\\
 Le premier th�or�me d'isomorphisme nous permet de conclure que ces deux groupes sont isomorphes. \\
La continuit� de l'application quotient $\overline{\varphi}$ vient de la d�finition de topologie quotient et montrons la continuit� de sa r�ciproque. \\
Soit $U$ un ouvert de $S^3 \times S^3 /\{\pm(1,1)\}$. On veut montrer que $\overline{\varphi}(U)$ est un ouvert de $\SO(4)$. \\
Par surjectivit� de $\pi$, $\overline{\varphi}(U)=\overline{\varphi}(\pi\circ\pi^{-1}(U))$ et donc $\overline{\varphi}(U)=\varphi(\pi^{-1}(U))$. Comme $\varphi$ est un diff�omorphisme local et que $\pi^{-1}(U)$ est un ouvert (par d�finition de la topoogie quotient) alors $\varphi(U)$ est un ouvert. \\
Le caract�re $\Ccal^\infty$ vient du fait que $\varphi$ est $\Ccal^\infty$ et que la diff�rentielle de $\varphi$ est inversible.
\end{proof}

Gr�ce � l'identification $S^3 \times S^3 /\{\pm(1,1)\}$ avec la quadrique projective $Q=\{ [u:v] \in \R P^7 | |u|=|v|\}$ (qui se fait en remarquant que $Q \ni [u:v]=\left[\frac{u}{|u|},\frac{v}{|v|} \right]$) et au r�sultat pr�c�dent alors $SO(4)$ s'identifie � $Q$ (comme vari�t� diff�rentiable. Comme, de plus, $O^-(4)$ s'identifie aussi � $\SO(4)$ (on prend un matrice $\gamma$ de $O^-(4)$ et on a $SO(4)=\gamma O^-(4)$) alors $Q$ contient deux familles $Q_\pm$ d'espaces projectifs  de dimension 3 (que l'on identifie � $\R P^3$) qui sont deux copies de $Q$. 





\ifwhole
 \end{document}
\fi