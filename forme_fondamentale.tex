
\newif\ifwhole

\wholetrue
% Ajouter \wholetrue si on compile seulement ce fichier

\ifwhole
 \documentclass[a4page,10pt]{article}
     \usepackage[Latin1]{inputenc}
\usepackage[francais]{babel}
\usepackage{amsmath,amssymb,amsthm}
\usepackage{textcomp}
\usepackage{mathrsfs}
\usepackage{algcompatible,algorithm  }
\usepackage[all]{xy}
\usepackage{hyperref}
\usepackage{fancyhdr}
\usepackage{supertabular}
\pagestyle{plain}

\toks0=\expandafter{\xy}
\edef\xy{\noexpand\shorthandoff{!?;:}\the\toks0 }
\makeatletter

\renewcommand*{\ALG@name}{Algorithme}

\makeatother

\renewcommand{\algorithmicrequire}{\textbf{\textsc {Entrées  :  } } }
\renewcommand{\algorithmicensure}{\textbf{\textsc { Sortie  :  } } }
\renewcommand{\algorithmicwhile}{\textbf{Tant que}}
\renewcommand{\algorithmicdo}{\textbf{faire }}
\renewcommand{\algorithmicif}{\textbf{Si}}
\renewcommand{\algorithmicelse}{\textbf{Sinon}}
\renewcommand{\algorithmicthen}{\textbf{alors }}
\renewcommand{\algorithmicend}{\textbf{fin}}
\renewcommand{\algorithmicfor}{\textbf{Pour}}
\renewcommand{\algorithmicuntil}{\textbf{Jusqu'à}}
\renewcommand{\algorithmicrepeat}{\textbf{Répéter}}

\newcommand{\A}{\mathbb{A}}
\newcommand{\B}{\mathbb{B}}
\newcommand{\C}{\mathbb{C}}
\newcommand{\D}{\mathbb{D}}
\newcommand{\E}{\mathbb{E}}
\newcommand{\F}{\mathbb{F}}
\newcommand{\G}{\mathbb{G}}
\renewcommand{\H}{\mathbb{H}}
\newcommand{\I}{\mathbb{I}}
\newcommand{\J}{\mathbb{J}}
\newcommand{\K}{\mathbb{K}}
\renewcommand{\L}{\mathbb{L}}
\newcommand{\M}{\mathbb{M}}
\newcommand{\N}{\mathbb{N}}
\renewcommand{\O}{\mathbb{O}}
\renewcommand{\P}{\mathbb{P}}
\newcommand{\Q}{\mathbb{Q}}
\newcommand{\R}{\mathbb{R}}
\renewcommand{\S}{\mathbb{S}}
\newcommand{\T}{\mathbb{T}}
\newcommand{\U}{\mathbb{U}}
\newcommand{\V}{\mathbb{V}}
\newcommand{\W}{\mathbb{W}}
\newcommand{\X}{\mathbb{X}}
\newcommand{\Y}{\mathbb{Y}}
\newcommand{\Z}{\mathbb{Z}}
\newcommand{\Acal}{\mathcal{A}}
\newcommand{\Bcal}{\mathcal{B}}
\newcommand{\Ccal}{\mathcal{C}}
\newcommand{\Dcal}{\mathcal{D}}
\newcommand{\Ecal}{\mathcal{E}}
\newcommand{\Fcal}{\mathcal{F}}
\newcommand{\Gcal}{\mathcal{G}}
\newcommand{\Hcal}{\mathcal{H}}
\newcommand{\Ical}{\mathcal{I}}
\newcommand{\Jcal}{\mathcal{J}}
\newcommand{\Kcal}{\mathcal{K}}
\newcommand{\Lcal}{\mathcal{L}}
\newcommand{\Mcal}{\mathcal{M}}
\newcommand{\Ncal}{\mathcal{N}}
\newcommand{\Ocal}{\mathcal{O}}
\newcommand{\Pcal}{\mathcal{P}}
\newcommand{\Qcal}{\mathcal{Q}}
\newcommand{\Rcal}{\mathcal{R}}
\newcommand{\Scal}{\mathcal{S}}
\newcommand{\Tcal}{\mathcal{T}}
\newcommand{\Ucal}{\mathcal{U}}
\newcommand{\Vcal}{\mathcal{V}}
\newcommand{\Wcal}{\mathcal{W}}
\newcommand{\Xcal}{\mathcal{X}}
\newcommand{\Ycal}{\mathcal{Y}}
\newcommand{\Zcal}{\mathcal{Z}}
\newcommand{\Ascr}{\mathscr{A}}
\newcommand{\Bscr}{\mathscr{B}}
\newcommand{\Cscr}{\mathscr{C}}
\newcommand{\Dscr}{\mathscr{D}}
\newcommand{\Escr}{\mathscr{E}}
\newcommand{\Fscr}{\mathscr{F}}
\newcommand{\Gscr}{\mathscr{G}}
\newcommand{\Hscr}{\mathscr{H}}
\newcommand{\Iscr}{\mathscr{I}}
\newcommand{\Jscr}{\mathscr{J}}
\newcommand{\Kscr}{\mathscr{K}}
\newcommand{\Lscr}{\mathscr{L}}
\newcommand{\Mscr}{\mathscr{M}}
\newcommand{\Nscr}{\mathscr{N}}
\newcommand{\Oscr}{\mathscr{O}}
\newcommand{\Pscr}{\mathscr{P}}
\newcommand{\Qscr}{\mathscr{Q}}
\newcommand{\Rscr}{\mathscr{R}}
\newcommand{\Sscr}{\mathscr{S}}
\newcommand{\Tscr}{\mathscr{T}}
\newcommand{\Uscr}{\mathscr{U}}
\newcommand{\Vscr}{\mathscr{V}}
\newcommand{\Wscr}{\mathscr{W}}
\newcommand{\Xscr}{\mathscr{X}}
\newcommand{\Yscr}{\mathscr{Y}}
\newcommand{\Zscr}{\mathscr{Z}}
\newcommand{\Afrak}{\mathfrak{A}}
\newcommand{\Bfrak}{\mathfrak{B}}
\newcommand{\Cfrak}{\mathfrak{C}}
\newcommand{\Dfrak}{\mathfrak{D}}
\newcommand{\Efrak}{\mathfrak{E}}
\newcommand{\Ffrak}{\mathfrak{F}}
\newcommand{\Gfrak}{\mathfrak{G}}
\newcommand{\Hfrak}{\mathfrak{H}}
\newcommand{\Ifrak}{\mathfrak{I}}
\newcommand{\Jfrak}{\mathfrak{J}}
\newcommand{\Kfrak}{\mathfrak{K}}
\newcommand{\Lfrak}{\mathfrak{L}}
\newcommand{\Mfrak}{\mathfrak{M}}
\newcommand{\Nfrak}{\mathfrak{N}}
\newcommand{\Ofrak}{\mathfrak{O}}
\newcommand{\Pfrak}{\mathfrak{P}}
\newcommand{\Qfrak}{\mathfrak{Q}}
\newcommand{\Rfrak}{\mathfrak{R}}
\newcommand{\Sfrak}{\mathfrak{S}}
\newcommand{\Tfrak}{\mathfrak{T}}
\newcommand{\Ufrak}{\mathfrak{U}}
\newcommand{\Vfrak}{\mathfrak{V}}
\newcommand{\Wfrak}{\mathfrak{W}}
\newcommand{\Xfrak}{\mathfrak{X}}
\newcommand{\Yfrak}{\mathfrak{Y}}
\newcommand{\Zfrak}{\mathfrak{Z}}
\newcommand{\afrak}{\mathfrak{a}}
\newcommand{\bfrak}{\mathfrak{b}}
\newcommand{\cfrak}{\mathfrak{c}}
\newcommand{\dfrak}{\mathfrak{d}}
\newcommand{\efrak}{\mathfrak{e}}
\newcommand{\ffrak}{\mathfrak{f}}
\newcommand{\gfrak}{\mathfrak{g}}
\newcommand{\hfrak}{\mathfrak{h}}
\newcommand{\ifrak}{\mathfrak{i}}
\newcommand{\jfrak}{\mathfrak{j}}
\newcommand{\kfrak}{\mathfrak{k}}
\newcommand{\lfrak}{\mathfrak{l}}
\newcommand{\mfrak}{\mathfrak{m}}
\newcommand{\nfrak}{\mathfrak{n}}
\newcommand{\ofrak}{\mathfrak{o}}
\newcommand{\pfrak}{\mathfrak{p}}
\newcommand{\qfrak}{\mathfrak{q}}
\newcommand{\rfrak}{\mathfrak{r}}
\newcommand{\sfrak}{\mathfrak{s}}
\newcommand{\tfrak}{\mathfrak{t}}
\newcommand{\ufrak}{\mathfrak{u}}
\newcommand{\vfrak}{\mathfrak{v}}
\newcommand{\wfrak}{\mathfrak{w}}
\newcommand{\xfrak}{\mathfrak{x}}
\newcommand{\yfrak}{\mathfrak{y}}
\newcommand{\zfrak}{\mathfrak{z}}
\newcommand{\Abar}{\overline{A}}
\newcommand{\Bbar}{\overline{B}}
\newcommand{\Cbar}{\overline{C}}
\newcommand{\Dbar}{\overline{D}}
\newcommand{\Ebar}{\overline{E}}
\newcommand{\Fbar}{\overline{F}}
\newcommand{\Gbar}{\overline{G}}
\newcommand{\Hbar}{\overline{H}}
\newcommand{\Ibar}{\overline{I}}
\newcommand{\Jbar}{\overline{J}}
\newcommand{\Kbar}{\overline{K}}
\newcommand{\Lbar}{\overline{L}}
\newcommand{\Mbar}{\overline{M}}
\newcommand{\Nbar}{\overline{N}}
\newcommand{\Obar}{\overline{O}}
\newcommand{\Pbar}{\overline{P}}
\newcommand{\Qbar}{\overline{Q}}
\newcommand{\Rbar}{\overline{R}}
\newcommand{\Sbar}{\overline{S}}
\newcommand{\Tbar}{\overline{T}}
\newcommand{\Ubar}{\overline{U}}
\newcommand{\Vbar}{\overline{V}}
\newcommand{\Wbar}{\overline{W}}
\newcommand{\Xbar}{\overline{X}}
\newcommand{\Ybar}{\overline{Y}}
\newcommand{\Zbar}{\overline{Z}}
\newcommand{\abar}{\overline{a}}
\newcommand{\bbar}{\overline{b}}
\newcommand{\cbar}{\overline{c}}
\newcommand{\dbar}{\overline{d}}
\newcommand{\ebar}{\overline{e}}
\newcommand{\fbar}{\overline{f}}
\newcommand{\gbar}{\overline{g}}
\renewcommand{\hbar}{\overline{h}}
\newcommand{\ibar}{\overline{i}}
\newcommand{\jbar}{\overline{j}}
\newcommand{\kbar}{\overline{k}}
\newcommand{\lbar}{\overline{l}}
\newcommand{\mbar}{\overline{m}}
\newcommand{\nbar}{\overline{n}}
\newcommand{\obar}{\overline{o}}
\newcommand{\pbar}{\overline{p}}
\newcommand{\qbar}{\overline{q}}
\newcommand{\rbar}{\overline{r}}
\newcommand{\sbar}{\overline{s}}
\newcommand{\tbar}{\overline{t}}
\newcommand{\ubar}{\overline{u}}
\newcommand{\vbar}{\overline{v}}
\newcommand{\wbar}{\overline{w}}
\newcommand{\xbar}{\overline{x}}
\newcommand{\ybar}{\overline{y}}
\newcommand{\zbar}{\overline{z}}
\newcommand\bigzero{\makebox(0,0){\text{\huge0}}}
\newcommand{\limp}{\lim\limits_{\leftarrow}}
\newcommand{\limi}{\lim\limits_{\rightarrow}}


\DeclareMathOperator{\End}{\mathrm{End}}
\DeclareMathOperator{\Hom}{\mathrm{Hom}}
\DeclareMathOperator{\Vect}{\mathrm{Vect}}
\DeclareMathOperator{\Spec}{\mathrm{Spec}}
\DeclareMathOperator{\multideg}{\mathrm{multideg}}
\DeclareMathOperator{\LM}{\mathrm{LM}}
\DeclareMathOperator{\LT}{\mathrm{LT}}
\DeclareMathOperator{\LC}{\mathrm{LC}}
\DeclareMathOperator{\PPCM}{\mathrm{PPCM}}
\DeclareMathOperator{\PGCD}{\mathrm{PGCD}}
\DeclareMathOperator{\Syl}{\mathrm{Syl}}
\DeclareMathOperator{\Res}{\mathrm{Res}}
\DeclareMathOperator{\Com}{\mathrm{Com}}
\DeclareMathOperator{\GL}{\mathrm{GL}}
\DeclareMathOperator{\SL}{\mathrm{SL}}
\DeclareMathOperator{\SU}{\mathrm{SU}}
\DeclareMathOperator{\SO}{\mathrm{SO}}
\DeclareMathOperator{\Sp}{\mathrm{Sp}}
\DeclareMathOperator{\Spin}{\mathrm{Spin}}
\DeclareMathOperator{\Ker}{\mathrm{Ker}}
%\DeclareMathOperator{\Im}{\mathrm{Im}}

\headheight=0mm
\topmargin=-20mm
\oddsidemargin=-1cm
\evensidemargin=-1cm
\textwidth=18cm
\textheight=25cm
\parindent=0mm
\newif\ifproof
\newcommand{\demo}[1]{\ifproof #1 \else \fi}
 %Instruction d'utilisation : 
%les preuves du texte sont, en principe, entre des balises \demo, en sus des \begin{proof} pour l'instant.
%Laisser le texte tel quel, fait qu'elles ne sont pas affich�es.
%Mettre \prooftrue fait que toutes les preuves jusqu'� un \prooffalse ou la fin du document. 


 \begin{document}
\newtheorem{Thm}{Th�or�me}[section]
\newtheorem{Prop}[Thm]{Proposition}
\newtheorem{Propte}[Thm]{Propri�t�}
\newtheorem{Lemme}[Thm]{Lemme}
\newtheorem{Cor}[Thm]{Corollaire}


\theoremstyle{definition}

\newtheorem{Ex}[Thm]{Exemple}
\newtheorem{Def}[Thm]{D�finition}
\newtheorem{Defpropte}[Thm]{D�finition et propri�t�}
\newtheorem{Defprop}[Thm]{D�finition et proposition}
\newtheorem{Defthm}[Thm]{Th�or�me et d�finition}
\newtheorem{Not}[Thm]{Notation}
\newtheorem{Conv}[Thm]{Convention}
\newtheorem{Cons}[Thm]{Construction}

\theoremstyle{remark}
\newtheorem{Rq}[Thm]{Remarque}
\newtheorem{Slog}[Thm]{Slogan}
\newtheorem{Exo}[Thm]{Exercice}
\newcommand{\ff}{\prod \hspace{-0.42cm}\coprod}
\newcommand{\ffdiag}{\prod \hspace{-0.32cm}\coprod}
\fi

Soient $M$ une hypersurface de $\R P^n$, $p \in M$ et une �quation locale $f=0$ de $M$ au voisinage de $p$ ($f$ est une submersion en $p$ i.e. $d_pf \neq 0$). \\
En calculant dans les cartes de $\R P^n$, on peut voir que $T_pM$ co�ncide avec $T_p H$ o� $H$ est l'hyperplan $Ker(d_pf)$ auquel on a rajout� ses points � l'infini. \\
Comme $d_p f_{|H}=(d_p f)_{|T_p H}=0$ alors $p$ est un point critique de $f$ et donc on peut d�finir une hessienne sur $T_p H \times T_p H=T_p M \times T_p M$. On d�finit $\ff_p$ comme l'application faisant commuter le diagramme suivant : \\
  $\xymatrix{
    T_pM \times T_pM\ar[r]^{-H_pf}  \ar[rd]_{\ffdiag_p} & \R  \\
    & N_pM:=T_p \R P^n/T_pM \ar[u]_{\overline{d_pf}}
  }$ \\
	o� $\overline{d_pf}$ est l'isomorphisme induit par $d_pf$. \\
	Autrement dit, $\ff_p=-\overline{d_pf}^{-1} \circ H_pf$. \\
Si on change l'�quation locale de $M$ en $p$ en $g=0$ alors $\Ker(d_pf)=\Ker(d_pg)$ (car $f$ et $g$ correspondent au m�me hyperplan projectif $H$) et $d_pg=\frac{d_pg(x)}{d_pf(x)} d_pf$ o� $d_pf(x) \neq 0$ i.e. $d_pf$ et $d_pg$ ne diff�rent que d'un facteur multiplicatif. On en d�duit que la hessienne change du m�me coefficient (c.f. la formule de la hessienne). $\ff_p$ ne d�pend donc pas de l'�quation locale choisie. On l'appelle deuxi�me fondamentale de $S$ au point $p$. \\
On peut maintenant d�finir la structure pseudo-conforme projective de $M$ en $p$ comme l'ensemble des applications bilin�aires $\ff_p \circ \varphi$, o� $\varphi : N_pM \to \R$ est un isomorphisme (lin�aire). \\
On va �tudier l'exemple de l'hypersurface $M=\{(x,t) \in \R^n \times \R | t=q(x)\} \subset \R^{n+1} \subset \R P^{n+1}$ o� $q$ est une forme quadratique non d�g�n�r�e de $\R^n$ : \\
Soit $f: (x,t) \mapsto t-q(x)$. $f=0$ est une �quation globale de $M$. La hessienne de $f$ en tout point est $2b$ o� $b$ est la forme bilin�aire associ�e � $f$.
On peut identifier $M$ avec $\R^n$ gr�ce � la projection $(x,t) \mapsto x$ puis $N_pM$ avec $\{0_{\R^n}\} \times \R$. Avec cette identification, $\ff_p=(0,2b)$ en tout point $p$ de $M$. La structure pseudo-conforme projective de $M$ en tout point est $\{ (0,2\lambda b) | \lambda \in \R^*\}$. \\
On peut compactifier $M$ en ajoutant ses points � l'infini. On obtient alors la quadrique projective (lisse) $Q=\{ [s : x : t] | q(x)-st=0\}\subset \R P^{n+1}$ (par construction, $Q=\{[1 : x : t ] | q(x)=t  \} \cup \{[0 : x : 1 ] | q(x)=0  \}=M \cup \{[0 : x : 1 ] | q(x)=0  \}$). Cette compactification est conforme dans le sens o� il existe un isomorphisme $\varphi$ tel que le diagramme suivant commute : \\

$\xymatrix@!C{
    T_pM \times T_pM\ar[r]^{d_p i \times d_p i}  \ar[d]_{\ffdiag_p^M} & T_{i(p)}Q \times T_{i(p)}Q \ar[d]_{\ffdiag_p^Q}  \\
    N_pM \ar[r]^\varphi & N_{i(p)}Q
  }$ \\
	o� $i: M \hookrightarrow Q$ est l'inclusion et $\ff^M,\ff^Q$ sont les deuxi�mes formes fondamentales de $M$, $Q$.
	Ici, $\varphi$ est l'identit�. \\
	En r�sum�, on peut compactifier $\R^r \oplus \R^s$ muni de la m�trique $dx^2-dy^2$ en une quadrique projective lisse, dont la forme quadratique est de signature $(r+1,s+1)$, dans $\R P^{r+s+1}$ en conservant la structure pseudo-conforme projective.






\ifwhole
 \end{document}
\fi