
\newif\ifwhole

\wholetrue
% Ajouter \wholetrue si on compile seulement ce fichier

\ifwhole
 \documentclass[a4page,10pt]{article}
     \input{entete}
 \begin{document}
\input{enonce}
\fi

\begin{Lemme}\label{tril_nulle}
Soit $f : E^3 \to E$ une application trilin�aire sym�trique sur les deux premi�res variables (i.e. en $X$ et $Y$) et antisym�triques sur les deux derni�res (i.e. en $Y,Z$).
\end{Lemme}
\begin{proof}
Soit $x,y,z \in E$. \\
$f(x,y,z)=f(y,x,z)=-f(y,z,x)=-f(z,y,x)=f(z,x,y)=f(x,z,y)=-f(x,y,z)$. 
Et donc $f(x,y,z)=0$
\end{proof}
\begin{Lemme} \label{rang1}
Soient $X,Y,Z$ trois champs de vecteurs tangents de $S$ (vus comme des d�rivations). Alors, 
	\[\psi_+(XY\psi) \in \R \overline{\psi_-}
	\] 
	et 
	\[(XY\psi)\psi_- \in \R \overline{\psi_+}
\]
\end{Lemme}
\begin{proof}
Comme $V=\R \psi + Im(d\psi)$ est totalement isotrope et que $X\psi,Y \psi \in Im(d\psi)$ alors $\left\langle Y\psi,\psi \right\rangle=0$ et  $\left\langle Y\psi,X\psi \right\rangle=0$ . On en d�duit que $XY \psi$ est orthogonale � $\psi$.
En effet,$0=X \left\langle Y\psi,\psi \right\rangle$ puis par l'identit� de Leibniz, $\left\langle XY\psi,\psi \right\rangle+\underbrace{\left\langle Y\psi,X\psi \right\rangle}_{=0}=\left\langle XY\psi,\psi \right\rangle$.\\
De plus, comme $0=X\left\langle Y\psi,Z \psi \right\rangle=\left\langle XY\psi,Z \psi \right\rangle+\left\langle Y\psi,XZ \psi \right\rangle$ et que $\left\langle XY\psi,Z \psi\right\rangle-\left\langle YX\psi,Z \psi\right\rangle=\left\langle [X,Y]\psi,Z \psi\right\rangle=0$ alors $\left\langle XY \psi,Z \psi\right\rangle$ est trilin�aire, sym�trique en $X,Y$ et antisym�trique en $Y,Z$. D'apr�s le lemme \ref{tril_nulle}, $\left\langle XY \psi,Z \psi\right\rangle$ est nulle et donc $XY \psi$ est orthogonal avec $Z \psi$. \\
On en d�duit que que $XY \psi$ est orthogonale avec tout �l�ment de $V$ c'est-�-dire est � valeur dans $V^\perp=(\Ker(L_{\psi_+})  \cap Ker(R_{\psi_-})^\perp$. \\
En utilisant le fait que $(\Ker(f) \cap \Ker(g))^\perp=Im(f^*)+Im(g^*)$  pour des applications lin�aires $E \to F$ (o� $f^*$ est l'adjoint de $f$) et la proposition \ref{}, on obtient que : 
	\[ V^\perp=Im(L_{\psi_+}^*)+Im(R_{\psi_-}^*)=Im(L_{\overline{\psi_+}})+Im(R_{\overline{\psi_-}})=\overline{\psi_+} \Ocal+\Ocal \overline{\psi_-}
\]
Puis, en utilisant la proposition \ref{} (car $\psi_-\psi_+=0$) et le fait que $N(\psi_+)=N(\psi_-)=0$, on d�duit que $\psi_+(XY\psi) \in \R \overline{\psi_-}$ et $(XY\psi)\psi_- \in \R \overline{\psi_+}$. \\

\end{proof}


\begin{Thm}
$(d \psi_-) \psi_+=\psi_- d \psi_+=0$
\end{Thm}

\begin{proof}
Soit $X$ un champ de vecteurs sur $S$ et montrons que $\psi_- X \psi_+$ s'annule en tout point $p \in S$. \\
Le cas o� $X \psi_+(p) =0$ est trivial, on peut donc supposer que $X \psi_+(p) \neq 0$. \\
Soit $Y$ un champ de vecteur ne s'annulant pas en $p$ tel que $\psi_+(p) (XY\psi)(p)=0$. \\
D'apr�s les identit�s \ref{}, on sait que $\psi_+ (Y \psi)=0$ et donc en d�rivant cette identit� par $X$ en $p$, on obtient : 
	\[0=X (\psi_+ (Y\psi))(p)=X\psi_+(p)(Y\psi)(p)+\psi_+(p)(XY\psi)(p)=X\psi_+(p)(Y\psi)(p)
\]
Ainsi, $\psi(p)$ et $Y\psi(p)$ sont dans le noyau $\Ker(L_{X\psi_+(p)})$ mais aussi dans $\Ker(R_{\psi_-(p)})$. On en d�duit que leur intersection est de dimension 3 (car de dimension impaire c.f. \ref{}) et donc $\psi_-(p)X\psi_+(p)=0$.
\end{proof}



\ifwhole
 \end{document}
\fi