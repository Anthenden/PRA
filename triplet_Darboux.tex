
\newif\ifwhole

\wholetrue
% Ajouter \wholetrue si on compile seulement ce fichier

\ifwhole
 \documentclass[a4page,10pt]{article}
     \input{entete}
 \begin{document}
\input{enonce}

\fi

\begin{Thm}
Soit $E$ un espace de dimension 2 et $L_1 : E \to \R^3$ une application lin�aire injective. Soit $L_2 : E \to \R^3$ une application lin�aire telle que $\forall u \in E,  L_1(u) \cdot L_2(u) =0$. Alors il existe un unique $h \in \R^3$ tel que $L_2=h \times L_1$ \;$(*)$
\end{Thm}

\begin{proof}
Soit $\Bcal=(w_1=L_1(v_1),w_2=L_1(v_2))$ une base orthonorm�e de $Im(L_1)$ (qui est de dimension 2 par le th�or�me du rang). Posons $u_1=L_2(v_1)$ et $u_2=L_2(v_2)$. \\
Trouver un $h$ v�rifiant $(*)$ est �quivalent � trouver un $h$ v�rifiant les �galit�s $\begin{cases}u_1=h \times w_1 \\ u_2=h \times w_2\end{cases}$.\\ En utilisant la premi�re �quation, on voit qu'un $h$ v�rifiant $(*)$ est de la forme $w_1 \times u_1+\lambda w_1$, $\lambda \in \R$ et gr�ce � la deuxi�me, $h$ est de la forme $w_2 \times u_2+\mu w_2$, $\mu \in \R$ (gr�ce � la formule du double produit vectoriel $\forall u,v,w \in \R^3, (u \times v) \times w=(u \cdot w) v-(v\cdot w) u$). 
Si $h$ est de ses deux formes, alors en faisant le produit scalaire avec $w_1$, on obtient que $\lambda=(w_1,w_2,u_2)$ o� $(\cdot,\cdot,\cdot)$ est le produit triple. \\
R�ciproquement, \\
$(w_1 \times u_1+\lambda w_1) \times w_2= w_1 \cdot w_2\; u_1-u_1\cdot  w_2\; w_1+\lambda w_1 \times w_2$.
On remarque ensuite que : \\
$0=L_1(v_1+v_2) \cdot L_2(v_1+v_2)=L_1(v_1) \cdot L_2(v_1)+L_1(v_2) \cdot L_2(v_2) +L_1(v_2) \cdot L_2(v_1)+L_1(v_1) \cdot L_2(v_2)=L_1(v_2) \cdot L_2(v_1)+L_1(v_1) \cdot L_2(v_2)$. \\
C'est-�-dire, $(w_1 \times u_1+\lambda w_1) \times w_2= w_1 \cdot w_2\; u_1+w_1\cdot  u_2\; w_1+\lambda w_1 \times w_2=u_2$
gr�ce � la d�composition de  $u_2$ dans la base $w_1,w_2,w_1 \times w_2$ (qui est orthonorm�e).
Les m�mes calculs nous montre que $\mu=(w_1,u_1,w_2)$ convient. \\
L'unicit� de $h$ d�coule du fait que l'ensemble de solution est l'intersection de deux droites non confondues (car $w_1 \neq w_2$).

\end{proof}



\ifwhole
 \end{document}
\fi